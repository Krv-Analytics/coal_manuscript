\setlength{\parindent}{20pt}
\section{Introduction}

%  ╭────────────────────────────────────╮
%  │  intro-intro                       |
%  ╰────────────────────────────────────╯

Reducing electricity sector emissions is crucial to decarbonizing the United States (U.S.) economy, particularly given electricity's growing importance for meeting transportation and building energy end uses. Accounting for 59\% of electricity greenhouse gas emissions but only 22\% of electricity generation, coal power remains the single largest source of CO\textsubscript{2} emissions in the U.S. electricity sector.
% \cite{us_epa_sources_2015}. 

For achieving decarbonization goals, the phaseout of coal power is as important as the build-out of renewable energy.
Burning coal has significant externalities and is responsible for the vast majority of criteria air pollutant emissions in the power sector.
% \cite{us_eia_where_nodate}. 
Coal-fired power plants have been linked to increased asthma deaths, mercury pollution, and higher hospitalization rates in affected communities.
% \cite{casey_coal-fired_2020}. 
Compared to natural gas, coal generation is relatively inflexible, reducing its value in power systems increasingly dominated by variable renewable energy. At the same time, aging infrastructure is increasing the costs associated with coal generation, while
% \cite{mac_kinnon_role_2018, hauenstein_us_2021, luderer_residual_2018}. 
the risk of stranded assets in the coal power sector remain low with less than 20\% of coal power infrastructure at risk.
% \cite{edwards_quantifying_2022, grubert_fossil_2020}. 
These characteristics and contexts all favor a fast and complete coal phaseout. 

%  ╭──────────────────────────────────────────────────────────────────╮
%  │  Intro Section 1: Not meeting climate targets                    |
%  ╰──────────────────────────────────────────────────────────────────╯
\subsection{Falling short of Climate Targets}
While the U.S. has seen a decline in coal electricity production over the past ten years, many studies indicate that coal power must be phased out entirely by 2035 if we are to meet existing climate goals, such as the global 1.5$^\circ$C target,\sidNote{Update these - new climate goals/policys have since come out}
% \cite{cui_uschina_2022, hultman_fusing_2020}, 
net-zero emissions, 
% \cite{larson_net-zero_2021}, 
and the Paris Climate Agreement. As shown in Fig.\ref{fig:ret_timeline}, relying on announced retirements is insufficient for meeting climate targets.
Even a best-case-scenario, enforcing retirements at the end of average coal plant anticipated lifespans (50 yrs old) only results in roughly 79\% of current coal capacity being retired through 2035. 
However, given that numerous coal plants have retirement plans extending well beyond the 50-year mark, achieving this scenario seems exceedingly challenging (see SI.\ref{fig:gen-ages} for a breakdown of the current coal fleet by age). 

An exclusive focus on age overlooks critical considerations that significantly impact a plant's viability and potential for continued operation. 
As evidenced by existing research and industry insights \textit{(cite papers here)}, as well as the above referenced age breakdown of the current US Coal Fleet,
factors such as technological advancements, retrofitting capabilities, environmental compliance, economic incentives, and regional energy demands all play pivotal roles in determining a coal plant's continued relevance (and operation beyond 50 years old).
There is a need for a more comprehensive, expedited coal phaseout plan. \sidTodo{need a stronger closing and better justification for both the 50 year old lifespan and plants retiring past this}

Despite numerous studies demonstrating the feasibility of phasing out coal by 2035 \textit{(cite papers here)}, 
real-world action remains elusive - thus far, feasibility has not translated into practice. \sidTodo{Add supporting statistics}
Rather than suggesting a feasible route forwards, we need to be evaluating the efficacy of strategies to speed up the retirement of US coal plants. 
Instead of proposing new policies, the primary focus of this paper is on understanding the nuances of the US coal fleet to better assess the effectiveness 
of existing policies and recommend optimal resource allocations. We aim to guide the targeting of these policies toward various aspects of the coal fleet.


%  ╭──────────────────────────────────────────────────────────────────╮
%  │  Figure 1: Timeline Figure                                       |
%  ╰──────────────────────────────────────────────────────────────────╯
    

    %  ╭────────────────────────────────────╮
%  │  timeline fig                      |
%  ╰────────────────────────────────────╯

\begin{figure}[htb]
    \centering
    \begin{minipage}{1\textwidth}
        \includesvg[inkscapelatex=false,width=\linewidth]{figs/svgs/fleetInfo/whatIf_retirement_timeline.svg}  
        \subcaption{What-If Retirement Timeline}
    \end{minipage}

    \medskip

    \begin{minipage}{1\textwidth}
    \begin{adjustbox}{width=\textwidth}
        \begin{small}
        \begin{tabular}{|l | r |}
            \hline
            \textbf{Metric} & \textbf{Value} \\
            \hline\hline
            Total Coal Capacity (as of the end of 2022) \sidTodo{add note on what qualifies as a coal plant} & 201.9 GW\\
            \hline
            Required Rate of Retirement per Year to Hit Target: 100\% of Coal Capacity Retired by 2035 & 16.82 GW/yr\\
            \hline
            Required Rate of Retirement per Year to Hit Target: 80\% of Coal Capacity Retired by 2035 & 13.46 GW/yr\\
            \hline
            Required Rate of Retirement per Year to Hit Target: 50\% of Coal Capacity Retired by 2035 & 8.41 GW/yr\\
            \hline
            Average Planned Rate of Retirement between 2023 and 2035 & 5.98 GW/yr\\
            \hline
        \end{tabular}
    \end{small}
    \end{adjustbox}
    \subcaption{Actual Retirement Target Requirements}
    \end{minipage}

    \caption{\textbf{Age-Based Retirement Scenario versus Current State of US Coal Retirements}}
    \medskip
    \footnotesize

    \textbf{(a)} A generator-level \textit{What-If} timeline of retiring US coal capacity, making the assumption that coal plants with no planned retirements will retire at 50 years old (see SI.\ref{fig:gen-ages} for an understanding of why this what-if scenario is not realistic).
    The 50 year old retirement age comes from the average age of retirement of a coal plant in the US, and is supported by Grubert's assertion that most coal capacity will retire when it reaches the age of 50...
    Regardless, this still leaves us shy of climate targets and hinges very heavily on a large amount of coal capacity retiring very quickly.
    \textbf{(b)} A breakdown of retirement rates needed to hit climate targets versus the current rate of coal retirements in the US.

    \label{fig:ret_timeline}
\end{figure}


\subsubsection*{Centralizing Retirement Strategy}
\stuTodo{Introduce manifold learning method here. } 



%  ╭──────────────────────────────────────────────────────────────────╮
%  │  Intro Section 2: Why this is a hard/interesting problem         |
%  ╰──────────────────────────────────────────────────────────────────╯
\subsection{The Complexities of Coal Plant Retirement}

\subsubsection{Drawbacks of Vulnerability Scoring and Age-centric Analyses}

Many studies (insert citations here) point to technical, financial, and environmental 
coal plant attributes as a way to determine low hanging fruit and prioritize retirements, utilizing a composite vulnerability score that combines these factors to guide 
decision-making. In addition to the technical/environmental/financial vulnerability of a coal plant, there are social, health, and political perspectives that must be accounted for as well, just 
to name a few. However, even with a comprehensive dataset, a composite vulnerability score approach is an oversimplification of a complex problem.\stuNote{I think this can be combined with Section 1.3 and significantly shortened. With that said, we need a consensus on how to attack addressing coal plant retirement prediction. In my opinion, this is not the purpose of the paper (our method provides understanding of plants that are the most difficult to retire, but we need to decide that providing a "more accurate prediction" of retirement is something we are going after. There are numerous methods that our grouping will undoubtedly improve their predictions, but to what degree and why a particular method was chosen requires from scratch motivation. There is no way to objectively classify a GOOD and BAD prediction.}
First, by combining data into a single score, a significant amount of detail is lost and nuances/interdependence's are rolled up into a subjective 
ranking. Different coal plants might operate in diverse contexts, locations, and regulatory environments. What is considered important in 
one scenario might not hold the same weight in another. Second, deciding how much weight to assign to each factor in the composite score is subjective. Different stakeholders may have 
different opinions on the importance of various components. Third, a composite score lacks detail and interpretability. Does a high vulnerability score mean a coal plant is 
overall vulnerable, or does it excel in only one aspect while lagging in others?
\subsubsection{\textit{the Curse of Dimensionality}}

\jezTodo{Manifold Learning and Multiverse Analysis!}
The coal plant retirement problem suffers from \textit{the curse of dimensionality}, where as the dimensions of a dataset 
increase, the number of data points needed to guarantee reliable results grows exponentially. In the case of the current US coal 
fleet, we have a very limited number of coal plants with very high-dimensional data associated with each. As such, standard 
unsupervised methods fail to reliably predict which active coal plants have planned retirements on the books. Additionally, the 
current arsenal of artificial intelligence methods fall short when it comes to comparing and analyzing complex objects across different 
data disciplines. In regards to coal plants, we aim to analyze Environmental, Technical, Financial, Political, and Health data, making 
it again difficult to understand the complex interactions between all variables at a high level. 

Even advanced deep learning algorithms, 
such as Gradient Boosting, struggle to effectively overcome the curse of dimensionality and provide accurate results. When looking to 
understand the variables that are important in grouping coal plants and the important variable determinants of planned retirement, different deep learning methods give us 
different results, showing how complex the coal plant retirement problem really is (\textit{See appendix for explanatory figs SI.\ref{fig:ML-featureImportance}, \ref{fig:ML-Predictions}}).


\subsubsection{Note on a Changing Retirement Landscape}\label{subsec:changing-retirement-landscape}

\stuNote{On condensing these three subsections - I think a lot of this info speaks to \textit{Why this problem is hard} which needs to stated as soon as possible (like right after saying what we did and why it is important - before introduction of method)}

The evolution of coal plant retirements has undergone a significant transformation. 
In the past, retirements often occurred organically, primarily driven by internal financial considerations, 
changing market dynamics, or technological obsolescence \textit{insert citation}. 
These retirements were driven by the economic viability of coal plants and their ability to compete 
in the energy market. Such retirements were not directly tied to explicit environmental or climate goals. 
However, the contemporary landscape is marked by a paradigm shift in the rationale behind coal plant retirements. 
The urgency to address climate change, reduce greenhouse gas emissions, and transition toward cleaner energy 
sources has become a driving force behind retirement decisions. This shift is propelled by the recognition of 
the critical role of coal-fired power generation in contributing to global carbon emissions and exacerbating 
climate-related challenges.

While past retirements might offer insights into operational challenges and procedural aspects of phasing out 
coal plants, they do not serve as reliable indicators for predicting future retirements. Several reasons underscore 
this departure from past trends. First, the current impetus for coal phaseout arises from external factors like 
international climate agreements (e.g., the Paris Agreement), increased public awareness of environmental issues, 
and regulatory frameworks that prioritize emission reductions. These factors introduce new dynamics that were 
absent or less pronounced during the organic retirement era. Second, decisions to retire coal plants are now embedded 
in a complex web of considerations, including economic viability, technological feasibility, regulatory compliance, 
community impacts, and climate objectives. This multi-dimensional decision landscape surpasses the relatively simpler 
determinants of past retirements. The involvement of diverse stakeholders, from local communities to environmental 
organizations, has gained prominence in modern coal phaseout initiatives. Ensuring a just transition for affected 
communities and addressing social equity concerns were not as central to past retirements. Third, there are new factors playing
a role in coal plant lifecycles that did not exist a few years ago. For example, crypto-currency has breathed life into dying coal plants. 
Through both PPAs and increased grid loads, server farms primarily used to 
mine bit-coin and other blockchain based currencies have began purchasing cheap power from coal plants on the brink, hungry 
for alternative or additional sources of revenue. \textit{Need citations here, heard about this through work - will need 
to get more info and make this punchier}

The shift from organic retirements to strategic retirements, motivated by climate and policy objectives as well as declining renewable
energy prices and government subsidies, introduces complexities 
that make the application of supervised machine learning challenging in the context of predicting future coal plant retirements. 
Training on data that includes retired coal plants fails to capture the transformational shift in decision drivers. The need for 
creative retirement solutions continues to be of paramount importance - assuming coal plants will retire on their own is overly 
optimistic, neglecting the pressing need for a transition to clean energy.

\sidTodo{This is just a dimensionality rediction. This section should probably be reworked to address the \textbf{why not kmeans} question I posed below this section... We need to: 

1 - why is our dimensionality reduction better / different?

2 - how do multiple perspectives play into this?

3 - not sure the following paragraph is necessary - can be shortened and reworked/added to the second paragraph addressing age in \textit{Coal is Not Being Retired Fast Enough to Meet Climate Targets}
}

While using vulnerability scores in the context of scenario modeling helps to compare the potential impacts of different phase out strategies and 
optimize an overall retirement plan, this approach merely produces a prescriptive list that can either be followed or deviated from - without giving 
the insights required to pursue coal plant retirements on a realistic scale. Additionally, the idea of an \textit{optimal} retirement plan does not work well 
within the political context of the United States where extraneous reasons can lead coal plants ready for retirement to continue operating long 
past their prime and/or plants in good economic standing to be shut down early. By delving into these multifaceted dimensions, our study aims to provide a 
more comprehensive and nuanced perspective, shedding light on the inadequacies of age-centric analyses and offering a more holistic 
framework for evaluating the future of coal plants in the evolving energy landscape.


%  ╭──────────────────────────────────────────────────────────────────╮
%  │  Intro Section 3: Our Contributions                              |
%  ╰──────────────────────────────────────────────────────────────────╯

\subsection{A Novel Approach to Coal Retirement}

\subsubsection{Centralizing Retirement Strategy}

To comprehensively address the multifaceted challenges associated with the US Coal Plant Fleet, it is imperative to group coal plants based on various dimensions, 
encompassing environmental, economic, and operational considerations, as well as political landscapes, public opinion, health, and localized climate targets. Clustering allows us to discern patterns and relationships within the vast and diverse dataset, 
enabling a more nuanced understanding of the impacts associated with coal plant retirement. By considering these multifarious facets, grouping plants ensures a more 
holistic and informed strategy for the centralization of coal retirement efforts.

To date there have been no wholistic attempts to classify the US Coal Plant Fleet from an interdisciplinary perspective. Given the intricate nature of coal 
plants as multifunctional entities with diverse characteristics, clustering becomes essential for capturing and quantifying the nuanced similarities among these 
complex objects (coal plants). Traditional classification methods fall short in handling the intricacies inherent in coal plants, but the combination of manifold learning and topological data analysis offer 
a nuanced perspective, allowing us to identify groups of plants that share commonalities in terms of age, emissions, operational efficiency, and other relevant factors. 
This nuanced understanding is crucial for devising a centralized coal retirement strategy that accounts for the diversity and complexity inherent in the coal fleet.

The need for expediency in devising a coal retirement strategy is paramount in addressing pressing environmental and economic concerns. As we aim to expedite retirements across the country, 
a multi-versal classification of coal facilities along every single facet of our dataset accelerates the allocation of resources to targeted groups, thereby hastening the phased retirement of 
coal power plants. This approach ensures a more agile response to the evolving dynamics of the US Coal Plant Fleet, helping to prompt a timely and effective centralization of early retirement efforts.

Why not rely on past coal plant retirements to forecast future patterns and strategy? A detailed explanation is provided in SI.\ref{subsec:changing-retirement-landscape}. 
That said, rapid shifts in clean energy pricing, government subsidies, utility and state-level climate goals, CO\textsubscript{2} reduction initiatives, and air quality 
compliance measures have altered the landscape significantly. The dynamics of previous coal plant retirements no longer accurately represent our current context.


\subsubsection{Manifold Learning Title}\stuNote{Advantages and the Downfall of Manifold learning = we introduce multiverse analysis. This is where we highlight the ability to consider a distribution of graph models, and our selection criteria.}
We aim to acknowledge the transformational nature of coal phaseout motivations and propose an analytical framework that reflects the 
dynamic and multidimensional factors influencing coal plant retirement decisions. As such, we propose a novel method of analyzing the 
US coal plant space to provide new and actionable insight. Here is what our methodology adds to the academic discipline:

\begin{enumerate}
    \item \textbf{Contextual Flexibility:} By not relying on predefined composite scores or fixed weights, our method can adapt to different scenarios, regulatory environments, and stakeholder perspectives.
    \item \textbf{Holistic Understanding:} By allowing for a more nuanced and comprehensive understanding of a coal plant's/group of coal plants' performance, across various dimensions, we enable stakeholders to make more informed decisions.
    \item \textbf{Transparency:} Our approach makes the evaluation process more transparent by presenting data and relationships directly. This transparency can build trust among stakeholders, and help to avoid the interpretability issues inherent to composite scores.
    \item \textbf{Customization:} Different coal plants have unique characteristics. Our method can accommodate these variations, providing a more accurate representation of a group's strengths and weaknesses.
\end{enumerate}

We do not claim to have solved the coal plant retirement problem, nor to have figured out the best and fastest way to retire all coal plants. 
Instead, we bring a brand new perspective that gives policy makers and other stakeholders an in-depth understanding of the coal fleet from 
multiple perspectives.



