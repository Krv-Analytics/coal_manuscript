\setlength{\parindent}{20pt}


%  ╭────────────────────────────────────╮
%  │  intro-intro                       |
%  ╰────────────────────────────────────╯

\section{Introduction}


Reducing electricity sector emissions is crucial to decarbonizing the United States (U.S.) economy, particularly given electricity's growing importance for meeting transportation and building energy end uses.\stuNote{Note my field, but I think we come out the gates with the statement that we need a centralized strategy for Coal Plants. This feels like auxiliary info and we're warming up into paragraph four, which could maybe use some stronger language (see note 3). } Accounting for 59\% of electricity greenhouse gas emissions but only 22\% of electricity generation, coal power remains the single largest source of CO\textsubscript{2} emissions in the U.S. electricity sector.
% \cite{us_epa_sources_2015}. 

For achieving decarbonization goals, the phaseout of coal power is as important as the build-out of renewable energy.
Burning coal has significant externalities and is responsible for the vast majority of criteria air pollutant emissions in the power sector.
% \cite{us_eia_where_nodate}. 
Coal-fired power plants have been linked to increased asthma deaths, mercury pollution, and higher hospitalization rates in affected communities.
% \cite{casey_coal-fired_2020}. 
Compared to natural gas, coal generation is relatively inflexible, reducing its value in power systems increasingly dominated by variable renewable energy. At the same time, aging infrastructure is increasing the costs associated with coal generation, while
% \cite{mac_kinnon_role_2018, hauenstein_us_2021, luderer_residual_2018}. 
the risk of stranded assets in the coal power sector remain low with less than 20\% of coal power infrastructure at risk.
% \cite{edwards_quantifying_2022, grubert_fossil_2020}. 
These characteristics and contexts all favor a fast and complete coal phaseout. 

\subsubsection*{Coal is Not Being Retired Fast Enough to Meet Climate Targets}
While the U.S. has seen a decline in coal electricity production over the past ten years, many studies indicate that coal power must be phased out entirely by 2035 if we are to meet existing climate goals, such as the global 1.5$^\circ$C target,\sidNote{Update these - new climate goals/policys have since come out}
% \cite{cui_uschina_2022, hultman_fusing_2020}, 
net-zero emissions, 
% \cite{larson_net-zero_2021}, 
and the Paris Climate Agreement. As shown in Fig.\ref{fig:ret_timeline}, relying on announced retirements is insufficient for meeting climate targets.
Even a best-case-scenario, enforcing retirements at the end of average coal plant anticipated lifespans (50 yrs old) only results in roughly 79\% of current coal capacity being retired through 2035. 
However, given that numerous coal plants have retirement plans extending well beyond the 50-year mark, achieving this scenario seems exceedingly challenging (see SI.\ref{fig:gen-ages} for a breakdown of the current coal fleet by age). 

An exclusive focus on age overlooks critical considerations that significantly impact a plant's viability and potential for continued operation. 
As evidenced by existing research and industry insights \textit{(cite papers here)}, as well as the above referenced age breakdown of the current US Coal Fleet,
factors such as technological advancements, retrofitting capabilities, environmental compliance, economic incentives, and regional energy demands all play pivotal roles in determining a coal plant's continued relevance (and operation beyond 50 years old).
There is a need for a more comprehensive, expedited coal phaseout plan. \sidTodo{need a stronger closing and better justification for both the 50 year old lifespan and plants retiring past this}

Despite numerous studies demonstrating the feasibility of phasing out coal by 2035 \textit{(cite papers here)}, 
real-world action remains elusive - thus far, feasibility has not translated into practice. \textit{need statistic here} 
Rather than suggesting a feasible route forwards, we need to be evaluating the efficacy of strategies to speed up the retirement of US coal plants. 
Instead of proposing new policies, the primary focus of this paper is on understanding the nuances of the US coal fleet to better assess the effectiveness 
of existing policies and recommend optimal resource allocations. We aim to guide the targeting of these policies toward various aspects of the coal fleet.


%  ╭────────────────────────────────────╮
%  │  timeline fig                      |
%  ╰────────────────────────────────────╯

\begin{figure}[htb]
    \centering
    \begin{minipage}{1\textwidth}
        \includesvg[inkscapelatex=false,width=\linewidth]{svg_figs/fleetInfo/whatIf_retirement_timeline.svg}  
        \subcaption{What-If Retirement Timeline}
    \end{minipage}

    \medskip

    \begin{minipage}{1\textwidth}
    \begin{adjustbox}{width=\textwidth}
        \begin{small}
        \begin{tabular}{|l | r |}
            \hline
            \textbf{Metric} & \textbf{Value} \\
            \hline\hline
            Total Coal Capacity (as of the end of 2022) \sidTodo{add note on what qualifies as a coal plant} & 201.9 GW\\
            \hline
            Required Rate of Retirement per Year to Hit Target: 100\% of Coal Capacity Retired by 2035 & 16.82 GW/yr\\
            \hline
            Required Rate of Retirement per Year to Hit Target: 80\% of Coal Capacity Retired by 2035 & 13.46 GW/yr\\
            \hline
            Required Rate of Retirement per Year to Hit Target: 50\% of Coal Capacity Retired by 2035 & 8.41 GW/yr\\
            \hline
            Average Planned Rate of Retirement between 2023 and 2035 & 5.98 GW/yr\\
            \hline
        \end{tabular}
    \end{small}
    \end{adjustbox}
    \subcaption{Actual Retirement Target Requirements}
    \end{minipage}

    \caption{\textbf{Age-Based Retirement Scenario versus Current State of US Coal Retirements}}
    \medskip
    \footnotesize

    \textbf{(a)} A generator-level \textit{What-If} timeline of retiring US coal capacity, making the assumption that coal plants with no planned retirements will retire at 50 years old (see SI.\ref{fig:gen-ages} for an understanding of why this what-if scenario is not realistic).
    The 50 year old retirement age comes from the average age of retirement of a coal plant in the US, and is supported by Grubert's assertion that most coal capacity will retire when it reaches the age of 50...
    Regardless, this still leaves us shy of climate targets and hinges very heavily on a large amount of coal capacity retiring very quickly.
    \textbf{(b)} A breakdown of retirement rates needed to hit climate targets versus the current rate of coal retirements in the US.

    \label{fig:ret_timeline}
\end{figure}


\subsubsection*{Centralizing Retirement Strategy}
\stuTodo{Introduce manifold learning method here. } 


To comprehensively address the multifaceted challenges associated with the US Coal Plant Fleet, it is imperative to group coal plants based on various dimensions, 
encompassing environmental, economic, and operational considerations, as well as political landscapes, public opinion, and localized climate targets. Clustering allows us to discern patterns and relationships within the vast and diverse dataset, 
enabling a more nuanced understanding of the impacts associated with coal plant retirement. By considering these multifarious facets, grouping plants ensures a more 
holistic and informed strategy for the centralization of coal retirement efforts.

To date there have been no wholistic attempts to classify the US Coal Plant Fleet from an interdisciplinary perspective. Given the intricate nature of coal 
plants as multifunctional entities with diverse characteristics, clustering becomes essential for capturing and quantifying the nuanced similarities among these 
complex objects (coal plants). Traditional classification methods fall short in handling the intricacies inherent in coal plants, but the combination of manifold learning and topological data analysis offer 
a nuanced perspective, allowing us to identify groups of plants that share commonalities in terms of age, emissions, operational efficiency, and other relevant factors. 
This nuanced understanding is crucial for devising a centralized coal retirement strategy that accounts for the diversity and complexity inherent in the coal fleet.

The need for expediency in devising a coal retirement strategy is paramount in addressing pressing environmental and economic concerns. As we aim to expedite retirements across the country, 
a multi-versal clustering of facilitates along every single facet of our dataset accelerates the allocation of resources to targeted groups, thereby hastening the phased retirement of 
coal power plants. This approach ensures a more agile response to the evolving dynamics of the US Coal Plant Fleet, helping to prompt a timely and effective centralization of early retirement efforts.

Why not rely on past coal plant retirements to forecast future patterns and strategy? A detailed explanation is provided in SI.\ref{subsec:changing-retirement-landscape}. 
That said, rapid shifts in clean energy pricing, government subsidies, utility and state-level climate goals, CO\textsubscript{2} reduction initiatives, and air quality 
compliance measures have altered the landscape significantly. The dynamics of previous coal plant retirements no longer accurately represent our current context.

% A detailed understanding of the composition and
% diversity of the US coal fleet facilitates more targeted and thus effective policymaking, better resource allocation, and aids in-depth
% understandings of the barriers and incentives associated with retiring different facets of coal plants. Understanding coal plant retirement patterns 
% and guiding the distribution of resources for the phased retirement of coal power plants are crucial aspects we aim to address in this paper.\stuNote{I think we need to end this paragraph on a stronger note... ``it is imperative we elucidate an effective and coordinated centralized strategy"   (made clear from the dire situation of decarbonations from stats above). This is part 1 of motivating why clustering.}
% =======
% However, to date there have been no wholistic attempts to
% classify the US Coal Plant Fleet from an interdisciplinary perspective. Understanding coal plant retirement patterns 
% and guiding the distribution of resources for the phased retirement of coal power plants are crucial aspects we aim to address in this paper.


% A detailed understanding of the composition and
% diversity of the US coal fleet facilitates more targeted and thus effective policymaking, better resource allocation, and aids in-depth
% understandings of the barriers and incentives associated with retiring different facets of coal plants. 


    %  ╭────────────────────────────────────╮
    %  │  Drawbacks of Current Approaches   |
    %  ╰────────────────────────────────────╯

\subsubsection*{Vulnerability Scoring and Age-centric Analyses}

Many studies (insert citations here) point to technical, financial, and environmental 
coal plant attributes as a way to determine low hanging fruit and prioritize retirements, utilizing a composite vulnerability score that combines these factors to guide 
decision-making. In addition to the technical/environmental/financial vulnerability of a coal plant, there are social, health, and political perspectives that must be accounted for as well, just 
to name a few. However, even with a comprehensive dataset, a composite vulnerability score approach is an oversimplification of a complex problem.\stuNote{I think this can be combined with Section 1.3 and significantly shortened. With that said, we need a consensus on how to attack addressing coal plant retirement prediction. In my opinion, this is not the purpose of the paper (our method provides understanding of plants that are the most difficult to retire, but we need to decide that providing a "more accurate prediction" of retirement is something we are going after. There are numerous methods that our grouping will undoubtedly improve their predictions, but to what degree and why a particular method was chosen requires from scratch motivation. There is no way to objectively classify a GOOD and BAD prediction.}
First, by combining data into a single score, a significant amount of detail is lost and nuances/interdependence's are rolled up into a subjective 
ranking. Different coal plants might operate in diverse contexts, locations, and regulatory environments. What is considered important in 
one scenario might not hold the same weight in another. Second, deciding how much weight to assign to each factor in the composite score is subjective. Different stakeholders may have 
different opinions on the importance of various components. Third, a composite score lacks detail and interpretability. Does a high vulnerability score mean a coal plant is 
overall vulnerable, or does it excel in only one aspect while lagging in others?

While using vulnerability scores in the context of scenario modeling helps to compare the potential impacts of different phase out strategies and 
optimize an overall retirement plan, this approach merely produces a prescriptive list that can either be followed or deviated from - without giving 
the insights required to pursue coal plant retirements on a realistic scale. Additionally, the idea of an \textit{optimal} retirement plan does not work well 
within the political context of the United States where extraneous reasons can lead coal plants ready for retirement to continue operating long 
past their prime and/or plants in good economic standing to be shut down early. By delving into these multifaceted dimensions, our study aims to provide a 
more comprehensive and nuanced perspective, shedding light on the inadequacies of age-centric analyses and offering a more holistic 
framework for evaluating the future of coal plants in the evolving energy landscape.

% Other studies \textit{(cite papers here)} look to age as the primary determinant of coal plant retirement. While we intend to demonstrate the shortcomings of this approach 
% in the coming results, its worth prefacing that numerous factors contribute to the complexities of this decision-making process, and an exclusive focus on age overlooks 
% critical considerations that significantly impact a plant's viability and potential for continued operation. As evidenced by existing research and industry insights \textit{(cite papers here)},
% as well as an age breakdown of the current US Coal Fleet (see SI.\ref{fig:gen-ages}),
% factors such as technological advancements, retrofitting capabilities, environmental compliance, economic incentives, and regional energy demands all play pivotal roles 
% in determining a coal plant's continued relevance. 

    %  ╭────────────────────────────────────╮
    %  │  Why Cluster                       |
    %  ╰────────────────────────────────────╯


