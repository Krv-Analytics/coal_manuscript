\setlength{\parindent}{20pt}

%  ╭────────────────────────────────────╮
%  │  intro-intro                       |
%  ╰────────────────────────────────────╯

\section{Introduction}

The US Coal fleet represents a heterogeneous group of plants with unique sets of impacts. A detailed understanding of the composition and
diversity of the US coal fleet facilitates more targeted and thus effective policymaking, better resource allocation, and aids in-depth
understandings of the barriers and incentives associated with retiring different facets of coal plants.

Despite numerous studies demonstrating the feasibility of phasing out coal by 2035 \textit{(cite papers here)}, 
real-world action remains elusive - thus far, feasibility has not translated into practice. \textit{insert statistics about current phase out rate to prove this point} 
Rather than suggesting a feasible route forwards, we need to be evaluating the efficacy of strategies to speed up the retirement of US coal plants. 
Instead of proposing new policies, the primary focus of this paper is on understanding the nuances of the US coal fleet to better assess the effectiveness 
of existing policies and recommend optimal resource allocations. We aim to guide the targeting of these policies toward various aspects of the coal fleet.

    %  ╭────────────────────────────────────╮
    %  │  Drawbacks of Current Approaches   |
    %  ╰────────────────────────────────────╯

\subsection{Vulnerability Scoring and Age-centric Analyses}

Retiring coal plants in the USA poses a complicated, multi-faceted problem. Many papers (insert citations here) point to technical, financial, and environmental 
coal plant attributes as a way to determine low hanging fruit and prioritize retirements, utilizing a composite vulnerability score that combines these factors to guide 
decision-making. In addition to the technical/environmental/financial vulnerability of a coal plant, there are social, health, and political perspectives as well, just 
to name a few. Even with a comprehensive dataset, a composite vulnerability score approach is an oversimplification of a complex problem. 
First, by combining data into a single score, a significant amount of detail is lost and nuances/interdependence's are rolled up into a subjective 
ranking. Different coal plants might operate in diverse contexts, locations, and regulatory environments. What is considered important in 
one scenario might not hold the same weight in another. Second, deciding how much weight to assign to each factor in the composite score is subjective. Different stakeholders may have 
different opinions on the importance of various components. Third, a composite score lacks detail and interpretability. Does a high vulnerability score mean a coal plant is 
overall vulnerable, or does it excel in only one aspect while lagging in others?

While using vulnerability scores in the context of scenario modeling helps to compare the potential impacts of different phase out strategies and 
optimize an overall retirement plan, this approach merely produces a prescriptive list that can either be followed or deviated from - without giving 
insight into the strategies required to pursue coal plants. Additionally, the idea of an \textit{optimal} retirement plan does not work well 
within the political context of the United States where extraneous reasons can lead coal plants ready for retirement to continue operating long 
past their prime and/or plants in good economic standing to be shut down early.

Other studies \textit{(cite papers here)} look to age as the primary determinant of coal plant retirement. While we intend to demonstrate the shortcomings of this approach 
in the coming results, its worth prefacing that numerous factors contribute to the complexities of this decision-making process, and an exclusive focus on age overlooks 
critical considerations that significantly impact a plant's viability and potential for continued operation. As evidenced by existing research and industry insights \textit{(cite papers here)}, 
factors such as technological advancements, retrofitting capabilities, environmental compliance, economic incentives, and regional energy demands all play pivotal roles 
in determining a coal plant's continued relevance. By delving into these multifaceted dimensions, our study aims to provide a more comprehensive and nuanced perspective, 
shedding light on the inadequacies of age-centric analyses and offering a more holistic framework for evaluating the future of coal plants in the evolving energy landscape.

    %  ╭────────────────────────────────────╮
    %  │  Why Cluster                       |
    %  ╰────────────────────────────────────╯


