\section{Introduction}

Despite numerous studies demonstrating the feasibility of phasing out coal by 2035 \textit{(cite papers here)}, 
real-world action remains elusive. Thus far, feasibility has not translated into practice. 
\textit{insert statistics about current phase out rate to prove this point} 
Rather than suggesting a feasible route forwards, we need to be suggesting strategies to speed up the retirement of US coal plants. 
We aim to highlight what's needed to retire coal plants and identify the essential policies required to achieve these ambitious goals.

% \begin{figure}[h]
%     % \includesvg[inkscapelatex=false,width=1\columnwidth]{svg_figs/fleetInfo/whatIf_retirement_timeline.svg}
%     \includesvg[inkscapelatex=false,width=1\columnwidth]{svg_figs/fleetInfo/gen_age.svg}
%     \includesvg[inkscapelatex=false,width=1\columnwidth]{svg_figs/fleetInfo/gen_ageAtRetirement.svg}
%     \label{fig:1}
%     \caption{\textbf{Current Coal Fleet Status - Age Breakdown} The figure illustrates the distribution of ages within the U.S. coal generator fleet, 
%     juxtaposed with their anticipated retirement ages. Notably, the average retirement age of plants that have announced planned retirement dates is 54.4 years, 
%     indicating a misalignment with imperative climate targets. Only 41.5\% (85.5 GW) of current coal capacity has announced plans to retire - with 91\% (78GW) 
%     of that capacity retiring by 2035.}
% \end{figure}

Retiring coal plants in the USA poses a complicated, multi-faceted problem. Many papers (insert citations here) point to technical, financial, and environmental 
coal plant attributes as a way to determine low hanging fruit and prioritize retirements, utilizing a composite vulnerability score that combines these factors to guide 
decision-making. In addition to the technical/environmental/financial vulnerability of a coal plant, there are social, health, and political perspectives as well, just 
to name a few. Even with a comprehensive dataset, a composite vulnerability score approach is an oversimplification of a complex problem:

\begin{itemize}
    \item By combining data into a single score, a significant amount of detail is lost and nuances/interdependence's are 
    rolled up into a subjective ranking. Different coal plants might operate in diverse contexts, locations, 
    and regulatory environments. What is considered important in one scenario might not hold the same weight in another.
    \item Deciding how much weight to assign to each factor in the composite score is subjective. Different stakeholders may have 
    different opinions on the importance of various components.
    \item A composite score lacks detail and interpretability. Does a high vulnerability score mean a coal plant is overall vulnerable, or does it excel in only one aspect while lagging in others?
\end{itemize}

While using vulnerability scores in the context of scenario modeling helps to compare the potential impacts of different phase out strategies and 
optimize an overall retirement plan, this approach merely produces a prescriptive list that can either be followed or deviated from - without giving 
insight into the strategies required to pursue coal plants. Additionally, the idea of an \textit{optimal} retirement plan does not work well 
within the political context of the United States where extraneous reasons can lead coal plants ready for retirement to continue operating long 
past their prime and/or plants in good economic standing to be shut down early. 