\setlength{\parindent}{20pt}

%  ╭────────────────────────────────────╮
%  │  intro-intro                       |
%  ╰────────────────────────────────────╯

\section{Introduction}



Reducing electricity sector emissions is crucial to decarbonizing the United States (U.S.) economy, particularly given electricity's growing importance for meeting transportation and building energy end uses.
% \cite{ipcc_global_2022}. 
Accounting for 59\% of electricity greenhouse gas emissions but only 22\% of electricity generation, coal power remains the single largest source of CO\textsubscript{2} emissions in the U.S. electricity sector.
% \cite{us_epa_sources_2015}. 

For achieving decarbonization goals, the phaseout of coal power is as important as the build-out of renewable energy.
Burning coal has significant externalities and is responsible for the vast majority of criteria air pollutant emissions in the power sector.
% \cite{us_eia_where_nodate}. 
Coal-fired power plants have been linked to increased asthma deaths, mercury pollution, and higher hospitalization rates in neighboring communities.
% \cite{casey_coal-fired_2020}. 
Compared to natural gas, coal generation is relatively inflexible, reducing its value in power systems increasingly dominated by variable renewable energy. At the same time, aging infrastructure is increasing the costs associated with coal generation.
% \cite{mac_kinnon_role_2018, hauenstein_us_2021, luderer_residual_2018}. 
Additionally, the risk of stranded assets in the coal power sector is significantly lower in the U.S. than in other countries, with less than 20\% of coal power infrastructure at risk.
% \cite{edwards_quantifying_2022, grubert_fossil_2020}. 
These characteristics and contexts all favor a fast and complete coal phaseout. 

While the U.S. has seen a decline in coal electricity production over the past ten years, many studies indicate that coal power must be phased out entirely by 2035 if we are to meet existing climate goals, such as the global 1.5$^\circ$C target, 
% \cite{cui_uschina_2022, hultman_fusing_2020}, 
net-zero emissions, 
% \cite{larson_net-zero_2021}, 
and the Paris Climate Agreement. As seen in Fig. \ref{fig:ret_timeline}, relying on announced retirements is insufficient for meeting climate targets.
Even enforcing retirements at the end of anticipated lifespans (50 yrs old) would result in an average of 12.95 GW/yr of coal capacity being retired through 2035,
not fast enough to eliminate even 80\% of US coal capacity by 2035. 
\textcolor{red}{\textit{This is plant level, not accounting for some of the Gens that are 50 shown in the below fig. Need to either reconcile this number or update the fig to be plant level and not generator level...}}
There is a need for a more comprehensive, expedited coal phaseout plan.

Despite numerous studies demonstrating the feasibility of phasing out coal by 2035 \textit{(cite papers here)}, 
real-world action remains elusive - thus far, feasibility has not translated into practice. \textit{insert statistics about current phase out rate to prove this point} 
Rather than suggesting a feasible route forwards, we need to be evaluating the efficacy of strategies to speed up the retirement of US coal plants. 
Instead of proposing new policies, the primary focus of this paper is on understanding the nuances of the US coal fleet to better assess the effectiveness 
of existing policies and recommend optimal resource allocations. We aim to guide the targeting of these policies toward various aspects of the coal fleet.
However, to date there have been no wholistic attempts to
classify the US Coal Plant Fleet from an interdisciplinary perspective. Understanding coal plant retirement patterns 
and guiding the distribution of resources for the phased retirement of coal power plants are crucial aspects we aim to address in this paper.

% A detailed understanding of the composition and
% diversity of the US coal fleet facilitates more targeted and thus effective policymaking, better resource allocation, and aids in-depth
% understandings of the barriers and incentives associated with retiring different facets of coal plants. 


%  ╭────────────────────────────────────╮
%  │  timeline fig                      |
%  ╰────────────────────────────────────╯

\begin{figure}[htb]
    \centering
    \begin{minipage}{1\textwidth}
        \includesvg[inkscapelatex=false,width=\linewidth]{svg_figs/fleetInfo/whatIf_retirement_timeline.svg}  
        \subcaption{What-If Retirement Timeline}
    \end{minipage}

    \begin{minipage}{1\textwidth}
    \begin{adjustbox}{width=\textwidth}
        \begin{small}
        \begin{tabular}{|l | r |}
            \hline
            \textbf{Metric} & \textbf{Value} \\
            \hline\hline
            Total Coal Capacity (as of the end of 2022) \textcolor{red}{\textit{add note on what qualifies as a coal plant}} & 201.9 GW\\
            \hline
            Required Rate of Retirement per Year to Hit Target: 100\% of Coal Capacity Retired by 2035 & 16.82 GW/yr\\
            \hline
            Required Rate of Retirement per Year to Hit Target: 80\% of Coal Capacity Retired by 2035 & 13.46 GW/yr\\
            \hline
            Required Rate of Retirement per Year to Hit Target: 50\% of Coal Capacity Retired by 2035 & 8.41 GW/yr\\
            \hline
            Average Planned Rate of Retirement between 2023 and 2035 & 5.98 GW/yr\\
            \hline
        \end{tabular}
    \end{small}
    \end{adjustbox}
    \subcaption{Actual Retirement Target Requirements}
    \end{minipage}

    \caption{\textbf{Age-Based Retirement Scenario versus Current State of US Coal Retirements}}
    \medskip
    \footnotesize

    \textbf{(a)} A generator-level \textit{What-If} timeline of retiring US coal capacity, making the assumption that coal plants with no planned retirements will retire at 50 years old (see SI.\ref{fig:gen-ages} for an understanding of why this what-if scenario is not realistic).
    The 50 year old retirement age comes from the average age of retirement of a coal plant in the US, and is supported by Grubert's assertion that most coal capacity will retire when it reaches the age of 50...
    Regardless, this still leaves us shy of climate targets and hinges very heavily on a large amount of coal capacity retiring very quickly.
    \textbf{(b)} A breakdown of retirement rates needed to hit climate targets versus the current rate of coal retirements in the US.

    \label{fig:ret_timeline}
\end{figure}


    %  ╭────────────────────────────────────╮
    %  │  Drawbacks of Current Approaches   |
    %  ╰────────────────────────────────────╯

\subsection{Vulnerability Scoring and Age-centric Analyses}

Retiring coal plants in the USA poses a complicated, multi-faceted problem. Many papers (insert citations here) point to technical, financial, and environmental 
coal plant attributes as a way to determine low hanging fruit and prioritize retirements, utilizing a composite vulnerability score that combines these factors to guide 
decision-making. In addition to the technical/environmental/financial vulnerability of a coal plant, there are social, health, and political perspectives that must be accounted for as well, just 
to name a few. However, even with a comprehensive dataset, a composite vulnerability score approach is an oversimplification of a complex problem. 
First, by combining data into a single score, a significant amount of detail is lost and nuances/interdependence's are rolled up into a subjective 
ranking. Different coal plants might operate in diverse contexts, locations, and regulatory environments. What is considered important in 
one scenario might not hold the same weight in another. Second, deciding how much weight to assign to each factor in the composite score is subjective. Different stakeholders may have 
different opinions on the importance of various components. Third, a composite score lacks detail and interpretability. Does a high vulnerability score mean a coal plant is 
overall vulnerable, or does it excel in only one aspect while lagging in others?

While using vulnerability scores in the context of scenario modeling helps to compare the potential impacts of different phase out strategies and 
optimize an overall retirement plan, this approach merely produces a prescriptive list that can either be followed or deviated from - without giving 
the insights required to pursue coal plant retirements on a realistic scale. Additionally, the idea of an \textit{optimal} retirement plan does not work well 
within the political context of the United States where extraneous reasons can lead coal plants ready for retirement to continue operating long 
past their prime and/or plants in good economic standing to be shut down early.

Other studies \textit{(cite papers here)} look to age as the primary determinant of coal plant retirement. While we intend to demonstrate the shortcomings of this approach 
in the coming results, its worth prefacing that numerous factors contribute to the complexities of this decision-making process, and an exclusive focus on age overlooks 
critical considerations that significantly impact a plant's viability and potential for continued operation. As evidenced by existing research and industry insights \textit{(cite papers here)},
as well as an age breakdown of the current US Coal Fleet (see SI.\ref{fig:gen-ages}),
factors such as technological advancements, retrofitting capabilities, environmental compliance, economic incentives, and regional energy demands all play pivotal roles 
in determining a coal plant's continued relevance. By delving into these multifaceted dimensions, our study aims to provide a more comprehensive and nuanced perspective, 
shedding light on the inadequacies of age-centric analyses and offering a more holistic framework for evaluating the future of coal plants in the evolving energy landscape.

    %  ╭────────────────────────────────────╮
    %  │  Why Cluster                       |
    %  ╰────────────────────────────────────╯


