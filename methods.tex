\section{Methods}

Here we give a description of data cleaning/preprocessing measures, the model selection 
process in an unsupervised learning setting, and finally a detailed analysis of the 
graph formation and structure. 


\subsection{Data Processing}

\stuTodo{Refer to figures for compilation of data sources}. 
Details we need to cover
\begin{itemize}
    \item data was scaled using standard scaler 
    \item categorical variables were one hot encoded 
    \item projected into two dimensions using UMAP (ignore hyperparameter selection, 
    it will be alluded to later)
\end{itemize}


\subsection{Model Generation}

\begin{itemize}
    \item Brief overview of how mapper works coverings + local HDBSCAN + node formation + edge formation
    \item Simplicial Complex $\to$ weighted graph.
    \item Topological meaning of being in the same node (or sharing an edge)
    \item Meaning of paths and connected components in a graph 
\end{itemize}


\subsection{Model Selection}
\jezTodo{ @sid Are we only imputing continuous variables? Will need your exact process
for the first few sections here.} 
\begin{itemize}
\item High variability in terms of model resolution
\item So we uniformly sample at random
from hyperparameter space. (UMAP, Kmapper cubical coverings and percent overlap.)
\item For each model we impute by sampling randomly from a normal distrubution based on
the existing column values (for continuous variables?).
\end{itemize}



\par For this specific problem, we are interested in models that split evenly over capacity but still 
provide interesting group profiles. We motivate this based on the distribution of the 
models that we generate. We want to select the elbow that balances the minimal
variance across group capacity sums while selecting a manageable number of groups. 
\jezTodo{@sid the x-axis (number of groups) and why we're minimizing needs to 
be motivated here from a Coal/policy development perspecitve.}

\begin{figure}[H]
\centering
\includesvg[inkscapelatex=false,width=\columnwidth]{figs/svgs/distribution-cracking.svg}
\subcaption{Cracking Distribution of Capacity (minimizing variance across groups)
}
\end{figure}
\stuNote{Write this subcaption, I would say without Cracking in the main title.
Don't think this is common terminology. }
\sidNote{Would be nice to circle the model we end up picking and to fit a curve to
aid in visualizing the elbow.}





 



 
\begin{itemize}
    \item Uniform sampling 
\end{itemize}