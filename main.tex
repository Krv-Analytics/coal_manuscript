% !TEX program = pdflatex

\documentclass{article}
\usepackage{svg}
\usepackage{subfiles}
\usepackage{hyperref}
\usepackage{hyperxmp}
\usepackage{qrcode}
\usepackage{float}

\usepackage[document]{ragged2e}
\usepackage{subcaption}

\usepackage{graphicx}
% \usetikzlibrary{shapes, positioning}
\usepackage{tikz}
\usepackage{xcolor}

\usepackage{geometry}
 \geometry{
 letterpaper,
 total={170mm,257mm},
 left=20mm,
 top=20mm,
 bottom=20mm,
 }

\definecolor{lightgrey}{RGB}{230,230,230}
\definecolor{bleu}{RGB}{ 49,140,231}

\newcommand{\highlight}[1]{%
  \fcolorbox{black}{lightgrey}{%
    \parbox{\dimexpr\linewidth-2\fboxsep-2\fboxrule}{%
    #1%
    }%
  }%
}

\hypersetup{
  colorlinks = true,
  urlcolor   = bleu,
  linkcolor  = bleu,
  citecolor  = bleu,
}

\title{Coal Manuscript - \textit{WIP draft}}
\author{}

% the following, to add affiliations with authors, is from: https://tex.stackexchange.com/a/604401/303504

\let\truehypersetup\hypersetup
\renewcommand\hypersetup[1]{}
\usepackage{bigfoot}
\let\hypersetup\truehypersetup

\DeclareNewFootnote{AAffil}[arabic]
\DeclareNewFootnote{ANote}[fnsymbol]

\usepackage{etoolbox}
\makeatletter
\patchcmd\maketitle{\def\@makefnmark{\rlap{\@textsuperscript{\normalfont\@thefnmark}}}}{}{}{}
\makeatother

% Hook into the \thanks command for the article class to print the footnotes
\makeatletter
\def\thanksAAffil#1{%
  \footnotemarkAAffil\protected@xdef\@thanks{\@thanks%
        \protect\footnotetextAAffil[\the \c@footnoteAAffil]{#1}}%
}
\def\thanksANote#1{%
  \footnotemarkANote%
  \protected@xdef\@thanks{\@thanks%
        \protect\footnotetextANote[\the \c@footnoteANote]{#1}}%
}
\makeatother

\date{}

\begin{document}

\maketitle


%  ╭──────────────────────────────────────────────────────────╮
%  │ Outline                                                  |
%  ╰──────────────────────────────────────────────────────────╯

\section{Intro Outline}

% Building a narrative (Stu's way ie thats how Alex does it) 
%  INTRODUCTION MUST ADRESS:
%  - why do we want to solve this problem (work towards a solution)
%  - why was it difficult to solve
%  - what have we achieved towards solving it 

\subsection{What is the problem and why do we want to solve it?}
\begin{itemize}
    % \item \textit{given our set of strategies, what is the best way to use them}
    % \item We leave new coal retirement strategies to our friends at the i.e. sierra club. Why aren't we tryint to make new strategies - because we have to work with what we have, do not have the ability to analyze the efficacy across teh whole coal fleet of things we havent seen before.
    % \item NOTE: We are not claiming to assign the \textit{best} strategy possible. Instead, given the set of strategies that have been noted as effective per a lit review, we are assigning them to the group in which they result in the highest efficacy.
    \item ASSUMPTION: Effective coal phase out strategies are a well known and fixed set (We need to research this set!).  
    \item \textbf{We want to minimize the total time of un-retired US coal plants given this fixed set of strategies} and an assumed upper bound on the resources allocated to doing so (Can we define this realistically? Like how much time/money/influence are we expected to get in the next X years... would be SICK to give an analysis that predicts how much we would need to actually hit that 2035 goal) 
    \item Motivate... why we don't expect to get unlimited resources to solve this problem and why the set of strategies is (somewhat) fixed 
    \item Coming up with the best assignment of strategies will decommission coal as quickly as possible, which can be motivated easily by $<$insert plethora of ES work on the subject $>$
    \item \textbf{We want to identify the most and least problematic sub-groups of coal plants with respect to this set of fixed strategies?}
    \item Allows us to allocate resources in an efficient manner (don't send a casual fisherman after Moby dick). We can recommend who to go after and who to avoid 
\end{itemize}


\subsection{Why is this problem hard to solve?}
\textcolor{red}{ARGUABLY THE MOST IMPORTANT PART OF THE INTRO}
 \textit{Predicting a strategy's efficacy for coal plant is HARD (Only ever seen a case by case?)! }
\begin{itemize}
    \item Variety of social, ... , factors impact a Coal plants behavior. Understanding its response requires understanding a coal plant and understanding similarity between coal plants
    \item We assume that similar coal plants will respond (with respect to and up until their similarity) similarly to a given strategy. \item Note that this work around is because we have NO historical data on how coal phase out strategy affects coal plant retirement (retirements B.C. are not helpful here) \textit{therefore we have to understand coal plants holistically before we can predict their responses}
    \item There are a variety of characters a coal plant could have to make it more or less difficult to effectively apply a strategy. On top of that, the particular relationships of a these characteristics are crucial in understanding strategy efficacy. 
\end{itemize}


\subsection{What have we achieved towards solving it?}
\textcolor{red}{The main point of this paper is convincing someone we've actually done these things. BUT, we are up a creek without a paddle if we cannot motivate to them what the problem is, why we want to solve it, and why it is difficult to do so. I honestly lose track of this myself quite often because the PROBLEM of Decarbonization is so prevalent, but we need to remember what WE are doing about it.}
\begin{itemize}
    \item For the set of coal plants, we prescribe an generalized assignment of strategies that approximately minimizes the total time of un-retired coal plants (assuming fixed resources)
    \item we have given a data-driven distance from retirement which allows us to proxy efficacy of a prescribed strategy. We can then identify the easy to retire and difficult to retire subgroups 
\end{itemize}

%  Alex prescription #2 

\section{What we have}
\begin{itemize}
    \item an explanation outlining why previously retired plants shouldn't be used to predict retirement now.
    \item a dataset with a spanning view of the problem (YOU NEED TO BE ABLE TO ARGUE THIS) 
    \item A clustering of coal plants over all columns of a dataset (groups that are similar according to the 'important' data) 
    \item A labelling of planned retired plants that gives a distance from retirement 
\end{itemize}

\section{What we Need}
\begin{enumerate}
     \item \textbf{A set of retirement strategies} (also the motivation of why this set is (somewhat) fixed) 
     \item Matching (not necessarily one-to-one) of strategies onto groups (MAIN RESULT 1)
     \item A set of most problematic and least problematic coal plants according to the set of strategies above (MAIN RESULT 2) 
     \item Memorable group names/labels based on most significant characteristics
     \item A cost/resource proxy for retirement difficulty (easier) 
     \item A timeline of estimated plant retirement according to path distance + an estimated end date (from literature/simulator) 
     \item an analysis of ownership for determining "problematic" group 
     \item WE NEED A WAY TO JUSTIFY THE NUMBER OF POLICY GROUPS SELECTED (the tool gives a lot of resolutions, but we are showing how to use the tool, not what the tool is)
     \item IDEA: Once we have the set of strategies, we pick the best resolutions
\end{enumerate}

\section{what we want}
\begin{itemize}
    \item A labelling of \textit{forced/early} planned retired plants that gives a distance from retirement  
    \item A customized timeline based on resource allocation 
    \item an analysis of the problematic group 
\end{itemize}


\section{After Getting what we need}

\begin{itemize}
    \item Paper Structure 
    \item what are going after in the intro/motivation? What figures do we need to communicate our point 
    \item JEREMY/STU: How are we introducing Manifold Learning + Hyperparameter Selection + graph representations in the intro?  
    \item JEREMY/STU: methods Section
    \item Dataset Breakdown for appendix
    
\end{itemize}


% \begin{itemize}
%     \item Motivation: \textcolor{green}{FLUFF}:coal is quite bad. Global warming is bad 
    
    
%     \textcolor{blue}{Better} Current analysis of phase out structures is lack luster.
    
%     \item Given the same strategies at our disposal right now, how do we use them more effectively? We are concerned with, given the resources you have (global scale or individual NGO level) how can you best use those resources? Objective: retire coal plants. Optimization: where do we focus our resources to achieve the objective.
%     \item The difficulty here is that analyzing coal plants to match corresponding phase-out strategies is a highly dimensional problem. \textbf{why is it difficult to analyze coal plants to predict how strategies will effect their retirement}
    
% \end{itemize}

% \textbf{\textit{we have a way to predict how effective a specific strategy will be at pushing a coal plant or group of plants to retire early}}

% \subsection{ \textbf{Why this problem is hard}}
% \begin{itemize}
%     \item you must look at relationships between different facets of data in order to understand a plant's behavior. e.g. Financial data will not give you insight into a plant's financial setting/future (ESG thing) you need to understand financial data in the context of political data, etc. to fully understand a plant.
%     \item Relationships between plants are valuable not to understand plants better, but to help understand how effective policy will be at targeting multiple plants (Sid's subtle point: historical data on coal plant retirement is not helpful, but current data following the Blah blah is?? This is not immediately understandable (by Stu) so we need to motive!) 
%     \item if we really understand coal plants, we can understand why certain strategies were effective. if we understand strategy efficacy, we can predict it.
%     \item we are not suggesting strategy, but if \textit{x} strategy works on \textit{a}, we can tell you how and why it work for \textit{b} based on complex similarities
% \end{itemize}


% \subsubsection{ \textbf{what have we achieved towards solving it, what we add to the field: }
% \begin{itemize}
%     \item we are bring a way to view the problem from a wholistic perspective, something that is distinctly lacking
%     \item we are allowing comparison not only based on attributes, but based on relationships to plant we know more / less about. 
%     \item why is it crucial to understand how/why all the different aspects react with / relate to eachother. 
% \end{itemize}


%  ╭──────────────────────────────────────────────────────────╮
%  │ Intro                                                    |
%  ╰──────────────────────────────────────────────────────────╯

\section{Introduction}

Despite numerous studies demonstrating the feasibility of phasing out coal by 2035 \textit{(cite papers here)}, real-world action remains elusive. Thus far, feasibility has not translated into practice. \textit{insert statistics about current phase out rate to prove this point} Rather than suggesting a feasible route forwards, we need to be suggesting strategies to speed up the retirement of US coal plants. We aim to highlight what's needed to retire coal plants and identify the essential policies required to achieve these ambitious goals.


\begin{figure}[h]
    % \includesvg[inkscapelatex=false,width=1\columnwidth]{svg_figs/fleetInfo/whatIf_retirement_timeline.svg}
    \includesvg[inkscapelatex=false,width=1\columnwidth]{svg_figs/fleetInfo/gen_age.svg}
    \includesvg[inkscapelatex=false,width=1\columnwidth]{svg_figs/fleetInfo/gen_ageAtRetirement.svg}
    \label{fig:1}
    \caption{\textbf{Current Coal Fleet Status - Age Breakdown} The figure illustrates the distribution of ages within the U.S. coal generator fleet, juxtaposed with their anticipated retirement ages. Notably, the average retirement age of plants that have announced planned retirement dates is 54.4 years, indicating a misalignment with imperative climate targets. Only 41.5\% (85.5 GW) of current coal capacity has announced plans to retire - with 91\% (78GW) of that capacity retiring by 2035.}
\end{figure}


Retiring coal plants in the USA poses a complicated, multi-faceted problem. Many papers (insert citations here) point to technical, financial, and environmental coal plant attributes as a way to determine low hanging fruit and prioritize retirements, utilizing a composite vulnerability score that combines these factors to guide decision-making. In addition to the technical/environmental/financial vulnerability of a coal plant, there are social, health, and political perspectives as well, just to name a few. Even with a comprehensive dataset, a composite vulnerability score approach is an oversimplification of a complex problem:

\begin{itemize}
    \item By combining data into a single score, a significant amount of detail is lost and nuances/interdependence's are rolled up into a subjective ranking. Different coal plants might operate in diverse contexts, locations, and regulatory environments. What is considered important in one scenario might not hold the same weight in another.
    \item Deciding how much weight to assign to each factor in the composite score is subjective. Different stakeholders may have different opinions on the importance of various components.
    \item A composite score lacks detail and interpretability. Does a high vulnerability score mean a coal plant is overall vulnerable, or does it excel in only one aspect while lagging in others?
\end{itemize}

While using vulnerability scores in the context of scenario modeling helps to compare the potential impacts of different phase out strategies and optimize an overall retirement plan, this approach merely produces a prescriptive list that can either be followed or deviated from - without giving insight into the strategies required to pursue coal plants. Additionally, the idea of an \textit{optimal} retirement plan does not work well within the political context of the United States where extraneous reasons can lead coal plants ready for retirement to continue operating long past their prime and/or plants in good economic standing to be shut down early. 


%  ╭──────────────────────────────────────────────────────────╮
%  │ Results                                                  |
%  ╰──────────────────────────────────────────────────────────╯

\input{FIG-CoalFleetPartition}


%  ╭──────────────────────────────────────────────────────────╮
%  │ EXTRAs - will be moved/deleted                           |
%  ╰──────────────────────────────────────────────────────────╯


\section{What about Deep Learning?}

\subsection{A Changing Retirement Landscape}

The evolution of coal plant retirements has undergone a significant transformation. 
In the past, retirements often occurred organically, primarily driven by internal financial considerations, 
changing market dynamics, or technological obsolescence \textit{insert citation}. 
These retirements were driven by the economic viability of coal plants and their ability to compete 
in the energy market. Such retirements were not directly tied to explicit environmental or climate goals. 
However, the contemporary landscape is marked by a paradigm shift in the rationale behind coal plant retirements. 
The urgency to address climate change, reduce greenhouse gas emissions, and transition toward cleaner energy 
sources has become a driving force behind retirement decisions. This shift is propelled by the recognition of 
the critical role of coal-fired power generation in contributing to global carbon emissions and exacerbating 
climate-related challenges.

While past retirements might offer insights into operational challenges and procedural aspects of phasing out 
coal plants, they do not serve as reliable indicators for predicting future retirements. Several reasons underscore 
this departure from past trends:

\begin{itemize}
    \item The current impetus for coal phaseout arises from external factors like international climate agreements (e.g., the Paris Agreement), increased public awareness of environmental issues, and regulatory frameworks that prioritize emission reductions. These factors introduce new dynamics that were absent or less pronounced during the organic retirement era.
    \item Decisions to retire coal plants are now embedded in a complex web of considerations, including economic viability, technological feasibility, regulatory compliance, community impacts, and climate objectives. This multi-dimensional decision landscape surpasses the relatively simpler determinants of past retirements.
    \item The involvement of diverse stakeholders, from local communities to environmental organizations, has gained prominence in modern coal phaseout initiatives. Ensuring a just transition for affected communities and addressing social equity concerns were not as central to past retirements.
    \item Recently, crypto-currency has also breathed life into dying coal plants. Through both PPAs and increased grid loads, server farms primarily used to mine bit-coin and other blockchain based currencies have began purchasing cheap power from coal plants on the brink, hungry for alternative or additional sources of revenue. \textit{Need citations here, heard about this through work - will need to get more info and make this punchier}
\end{itemize}

The shift from organic retirements to strategic retirements motivated by climate and policy objectives introduces complexities that make the application of supervised machine learning challenging in the context of predicting future coal plant retirements. Training on data that includes retired coal plants fails to capture the transformational shift in decision drivers. The need for creative retirement solutions continues to be of paramount importance - assuming coal plants will retire on their own is overly optimistic, neglecting the pressing need for a transition to clean energy.\\

\subsection{Coal is a High-Dimensional Problem}

The coal plant retirement problem suffers from \textit{the curse of dimensionality}, where as the dimensions of a dataset increase, the number of data points needed to guarantee reliable results grows exponentially. In the case of the current US coal fleet, we have a very limited number of coal plants with very high-dimensional data associated with each. As such, standard unsupervised methods fail to reliably predict which active coal plants have planned retirements on the books.

\begin{figure}[H]
    \includesvg[inkscapelatex=false,width=1\columnwidth]{svg_figs/ml_featureImportance_accuracy.svg}
    \includesvg[inkscapelatex=false,width=1\columnwidth]{svg_figs/ml_prediction_accuracy.svg}
    \label{fig:2}
    \caption{\textbf{Predicting Planned Retirements in the US Coal Plant Fleet} In the top figure, there is a clear difference between each method, with no clear consensus on a subset of variables that are important for predicting planned retirement. In the bottom figure, F1-Scores averaging 0.5 indicate that across 100 different runs, these models do not preform well in terms of their ability to balance precision and recall. 0.8 is considered the standard threshold at which F1-Scores indicate a model has preformed well. In other words, these methods poorly predict planned retirement and the key variables responsible for it.}
\end{figure}

Additionally, the current arsenal of artificial intelligence methods fall short when it comes to comparing and analyzing complex objects across different data disciplines. In regards to coal plants, we aim to analyze Environmental, Technical, Financial, Political, and Health data, making it again difficult to understand the complex interactions between all variables at a high level. Even advanced deep learning algorithms, such as Gradient Boosting, struggle to effectively overcome the curse of dimensionality and provide accurate results. When looking to understand the variables that are important in predicting whether a coal plant will retire, different deep learning methods give us different results, showing how complex the coal plant retirement problem really is.

\section{Combining Topology and Graphs}
\subsection{Advantages of a Novel Approach}
We aim to acknowledge the transformational nature of coal phaseout motivations and propose an analytical framework that reflects the dynamic and multidimensional factors influencing coal plant retirement decisions. As such, we propose a novel method of analyzing the US coal plant space to provide new and actionable insight. Here is what our methodology adds to the academic discipline:

\begin{itemize}
    \item \textbf{Contextual Flexibility:} By not relying on predefined composite scores or fixed weights, our method can adapt to different scenarios, regulatory environments, and stakeholder perspectives.
    \item \textbf{Holistic Understanding:} By allowing for a more nuanced and comprehensive understanding of a coal plant's/group of coal plants' performance, across various dimensions, we enable stakeholders to make more informed decisions.
    \item \textbf{Transparency:} Our approach makes the evaluation process more transparent by presenting data and relationships directly. This transparency can build trust among stakeholders, and help to avoid the interpretability issues inherent to composite scores.
    \item \textbf{Customization:} Different coal plants have unique characteristics. Our method can accommodate these variations, providing a more accurate representation of a group's strengths and weaknesses.
\end{itemize}

We do not claim to have solved the coal plant retirement problem, nor to have figured out the best and fastest way to retire all coal plants. Instead, we bring a brand new perspective that gives policy makers and other stakeholders an in-depth understanding of the coal fleet from multiple perspectives.

\subsection{Coal Graph Model}
Our goal is to map out the landscape of US Coal Plant Fleet. To that end, we generate a graph model of our data set that captures the relationships between all 269 coal plants in question with respect to the available data fields. The graph and resulting groups arise from the structure of our high dimensional data set, whose complex relationships are distilled into a graph model. Unlike traditional clustering methods, we use the graph model’s structure to build digestible descriptions of these groups. This allows us to illustrate which features connect the coal plants in each group, and compare group profiles across all available data fields.

The resulting model has 12 unique groupings of coal plants. While our groupings are derived from every feature of the data set, looking at homogeneous data fields within each group gives a high-level and digestible overview of why certain plants are grouped together. Here, we label the groups for increased interpretability:

\definecolor{col0}{HTML}{3B4CC0}
\definecolor{col1}{HTML}{6282ea}
\definecolor{col2}{HTML}{8db0fe}
\definecolor{col3}{HTML}{b9d0f9}
\definecolor{col4}{HTML}{dddcdc}
\definecolor{col5}{HTML}{f5c4ac}
\definecolor{col6}{HTML}{f4987a}
\definecolor{col7}{HTML}{dd5f4b}
\definecolor{col8}{HTML}{b40426}

\subsection{Coal Plant Groupings Breakdown}

\begin{figure}[H]
    \includesvg[inkscapelatex=false,width=1\columnwidth]{svg_figs/group_map.svg}
    \label{fig:2}
    \caption{\textbf{Plant Group Locations} Illustrates the locations of the plants in every group.}
\end{figure}

\subsection{Profiling Planned Coal Retirements}

\highlight{How to interpret the model below:
\begin{itemize}
    \item \textit{Nodes} (circles) represent clusters of similar coal plants, sized by the number of plants per cluster.
    \item \textit{Edges} (lines) connect similar clusters based on multidimensional relationships in the data.
    \item \textit{Groups} (connected nodes) make up connected components, or isolated sections of the graph. Here, we have focused on 3 of the 8 individual connected components that make up our graph of the entire US coal fleet.
\end{itemize}
}


% \begin{figure}[H]

%     \begin{minipage}{0.5\textwidth}
%         \includegraphics[width=\linewidth]{png_figs/shortest_path/group_1_vulnerability.png}
%         \subcaption{Group 1 Subgraph}
%     \end{minipage}%
%     \begin{minipage}{0.5\textwidth}
%         \includegraphics[width=\linewidth]{png_figs/shortest_path/group_2_vulnerability.png}
%         \subcaption{Group 2 Subgraph}
%     \end{minipage}
    
%     \begin{minipage}{0.5\textwidth}
%         \includegraphics[width=\linewidth]{png_figs/shortest_path/group_3_vulnerability.png}
%         \subcaption{Group 3 Subgraph}
%     \end{minipage}%
%     \begin{minipage}{0.4\textwidth}
%         \raggedleft
%         \includesvg[inkscapelatex=false,width=1\linewidth]{svg_figs/vulnerability_barPolar.svg}
%         \subcaption{Vulnerability Proxies per Group (by coal plant)}
%     \end{minipage}

%     \caption{\textbf{Proximity To Retirement Graphs} Nodes representing coal plants with the highest percentage of retiring capacity per plant are identified as target nodes. The edges in the graph are weighted to indicate the relationships between nodes. Employing Dijkstra's algorithm, we determine the shortest path and corresponding distance from these target nodes to every other node in the graph. This distance signifies the level of similarity; nodes with shorter paths to target nodes share more similar attributes with the retiring nodes. Conversely, nodes with longer distances have more dissimilar characteristics, indicating a greater divergence from the attributes of the retiring plants within nodes. \textit{Explain edge weighting here as well?}}
%     \label{fig:3}
% \end{figure}

\begin{figure}[H]

    \begin{minipage}{0.5\textwidth}
        \includegraphics[width=\linewidth]{png_figs/shortest_path/group_1_vulnerability.png}  
    \end{minipage}%
    \begin{minipage}{0.5\textwidth}
        \includegraphics[width=\linewidth]{png_figs/shortest_path/group_2_vulnerability.png}  
    \end{minipage}
    \begin{minipage}{0.5\textwidth}
        \includegraphics[width=\linewidth]{png_figs/shortest_path/group_3_vulnerability.png}  
    \end{minipage}%
    \begin{minipage}{0.4\textwidth}
        \raggedleft
        \includesvg[inkscapelatex=false,width=1\linewidth]{svg_figs/vulnerability_barPolar.svg}
    \end{minipage}
        
    \caption{\textbf{Proximity To Retirement Graphs} Nodes representing coal plants with the highest percentage of retiring capacity per plant are identified as target nodes. The edges in the graph are weighted to indicate the relationships between nodes. Employing Dijkstra's algorithm, we determine the shortest path and corresponding distance from these target nodes to every other node in the graph.
    % This is our approximation for a "coarse-grained" geodesic distance in high dimensional space
    This distance signifies the level of similarity; nodes with shorter paths to target nodes share more similar attributes with the retiring nodes. Conversely, nodes with longer distances have more dissimilar characteristics, indicating a greater divergence between the plants contained within them and the attributes of the retiring plants within target nodes. \textit{explain edge weighting here as well?}}
    \label{fig:3}
\end{figure}

% \begin{figure}[H]
%     \centering
%     \includegraphics[width=1\textwidth]{png_figs/shortest_path/group_1_vulnerability.png}  
%     \includegraphics[width=1\textwidth]{png_figs/shortest_path/group_2_vulnerability.png}  
%     \includegraphics[width=1\textwidth]{png_figs/shortest_path/group_3_vulnerability.png}  
%     %     \begin{tikzpicture}[overlay, remember picture, on background layer]
%     %     % Legend elements
%     %     \foreach \i in {0,1,2,3,4} {
%     %         \node [rectangle, fill=col\i, minimum size=0.3cm] at ({(\i+0.7)*2.5-7.5+0.3}, 0.5) {};
%     %         \node [right] at ({(\i+0.7)*2.5-7.2+0.3}, 0.5) {Group \i};}
%     %     \foreach \i in {5, 6, 7, 8} {
%     %         \node [rectangle, fill=col\i, minimum size=0.3cm] at ({(\i-3.8)*2.5-7.5+0.3}, -0.1) {};
%     %         \node [right] at ({(\i-3.8)*2.5-7.2+0.3}, -0.1) {Group \i};}
%     % \end{tikzpicture}
%     \caption{\textbf{Coal Fleet Graph} Each group has a unique mix of plant attributes across all categories. \textit{will need to find an additional model that highlights another interesting facet of the coal fleet}}
%     \label{fig:3}
% \end{figure}


\section{Supplemental Info/Figs}


% \begin{figure}
%     \includesvg[inkscapelatex=false,width=1\columnwidth]{svg_figs/correlation_matrix.svg}
%     \label{fig:s1}
%     \caption{\textbf{Variable Correlation Matrix} Note how the ret\_STATUS variable has a distinct lack of correlation to other variables}
% \end{figure}


\end{document}
