% !TEX program = pdflatex
% pdflatex -shell-escape main.tex

\documentclass{article}
\usepackage{svg}
\usepackage{subfiles}
\usepackage{hyperref}
\usepackage{hyperxmp}
\usepackage{qrcode}
\usepackage{float}

\usepackage{geometry}
\usepackage{tabularx, ltablex}
\usepackage{booktabs}
\usepackage{graphicx}
\usepackage{adjustbox}
\usepackage{float}

\usepackage[document]{ragged2e}
\usepackage{subcaption}

\usepackage{graphicx}
% \usetikzlibrary{shapes, positioning}
\usepackage{tikz}
\usepackage{xcolor}




%% Revising Macros for Sid, Grace, Stu, and Jez
\newcommand{\sidTodo}[1]{{\color{red}[Sid: {#1}]}}
\newcommand{\sidNote}[1]{\marginpar{\textcolor{red}{$\leftarrow$\fbox{Sid}}}\footnote{\sf {\color{red}Sid: {#1}}}}

\newcommand{\stuTodo}[1]{{\color{magenta}[Stu: {#1}]}}
\newcommand{\stuNote}[1]{\marginpar{\textcolor{magenta}{$\leftarrow$\fbox{Stu}}}\footnote{\sf {\color{magenta}Stu: {#1}}}}

\newcommand{\jezTodo}[1]{{\color{green}[Jez: {#1}]}}
\newcommand{\jezNote}[1]{\marginpar{\textcolor{green}{$\leftarrow$\fbox{Jez}}}\footnote{\sf {\color{green}Jez: {#1}}}}

\newcommand{\graceTodo}[1]{{\color{blue}[Grace: {#1}]}}
\newcommand{\graceNote}[1]{\marginpar{\textcolor{blue}{$\leftarrow$\fbox{Grace}}}\footnote{\sf {\color{blue}Grace: {#1}}}}


%  ╭────────────────────────────────────────────────────────────────────────────────────────╮
%  │  Imports required for table formatting -- see table_testing.tex for more info          |
%  ╰────────────────────────────────────────────────────────────────────────────────────────╯

% \usepackage{booktabs}
% \usepackage{caption}
% \usepackage{float}
% \usepackage{titlesec}
% \usepackage{capt-of}

% %dashed line
% \usepackage{array}
% \usepackage{arydshln}
% \setlength\dashlinedash{0.2pt}
% \setlength\dashlinegap{1.5pt}
% \setlength\arrayrulewidth{0.3pt}


\usepackage{geometry}
 \geometry{
 letterpaper,
 total={170mm,257mm},
 left=20mm,
 top=20mm,
 bottom=20mm,
 }

\definecolor{lightgrey}{RGB}{230,230,230}
\definecolor{bleu}{RGB}{ 49,140,231}

\newcommand{\highlight}[1]{%
  \fcolorbox{black}{lightgrey}{%
    \parbox{\dimexpr\linewidth-2\fboxsep-2\fboxrule}{%
    #1%
    }%
  }%
}

\hypersetup{
  colorlinks = true,
  urlcolor   = bleu,
  linkcolor  = bleu,
  citecolor  = bleu,
}

\title{Coal Manuscript - \textit{WIP draft}}
\author{}

% To add affiliations with authors, see: https://tex.stackexchange.com/a/604401/303504

\let\truehypersetup\hypersetup
\renewcommand\hypersetup[1]{}
\usepackage{bigfoot}
\let\hypersetup\truehypersetup

\DeclareNewFootnote{AAffil}[arabic]
\DeclareNewFootnote{ANote}[fnsymbol]

\usepackage{etoolbox}
\makeatletter
\patchcmd\maketitle{\def\@makefnmark{\rlap{\@textsuperscript{\normalfont\@thefnmark}}}}{}{}{}
\makeatother

% Hook into the \thanks command for the article class to print the footnotes
\makeatletter
\def\thanksAAffil#1{%
  \footnotemarkAAffil\protected@xdef\@thanks{\@thanks%
        \protect\footnotetextAAffil[\the \c@footnoteAAffil]{#1}}%
}
\def\thanksANote#1{%
  \footnotemarkANote%
  \protected@xdef\@thanks{\@thanks%
        \protect\footnotetextANote[\the \c@footnoteANote]{#1}}%
}
\makeatother

\date{}

\begin{document}

\maketitle


%  ╭──────────────────────────────────────────────────────────╮
%  │ Outline                                                  |
%  ╰──────────────────────────────────────────────────────────╯

% \section{Intro Outline}

% Building a narrative (Stu's way ie thats how Alex does it) 
%  INTRODUCTION MUST ADRESS:
%  - why do we want to solve this problem (work towards a solution)
%  - why was it difficult to solve
%  - what have we achieved towards solving it 

\subsection{What is the problem and why do we want to solve it?}
\begin{itemize}
    % \item \textit{given our set of strategies, what is the best way to use them}
    % \item We leave new coal retirement strategies to our friends at the i.e. sierra club. Why aren't we tryint to make new strategies - because we have to work with what we have, do not have the ability to analyze the efficacy across teh whole coal fleet of things we havent seen before.
    % \item NOTE: We are not claiming to assign the \textit{best} strategy possible. Instead, given the set of strategies that have been noted as effective per a lit review, we are assigning them to the group in which they result in the highest efficacy.
    \item ASSUMPTION: Effective coal phase out strategies are a well known and fixed set (We need to research this set!).  
    \item \textbf{We want to minimize the total time of un-retired US coal plants given this fixed set of strategies} and an assumed upper bound on the resources allocated to doing so (Can we define this realistically? Like how much time/money/influence are we expected to get in the next X years... would be SICK to give an analysis that predicts how much we would need to actually hit that 2035 goal) 
    \item Motivate... why we don't expect to get unlimited resources to solve this problem and why the set of strategies is (somewhat) fixed 
    \item Coming up with the best assignment of strategies will decommission coal as quickly as possible, which can be motivated easily by $<$insert plethora of ES work on the subject $>$
    \item \textbf{We want to identify the most and least problematic sub-groups of coal plants with respect to this set of fixed strategies?}
    \item Allows us to allocate resources in an efficient manner (don't send a casual fisherman after Moby dick). We can recommend who to go after and who to avoid 
\end{itemize}


\subsection{Why is this problem hard to solve?}
\textcolor{red}{ARGUABLY THE MOST IMPORTANT PART OF THE INTRO}
 \textit{Predicting a strategy's efficacy for coal plant is HARD (Only ever seen a case by case?)! }
\begin{itemize}
    \item Variety of social, ... , factors impact a Coal plants behavior. Understanding its response requires understanding a coal plant and understanding similarity between coal plants
    \item We assume that similar coal plants will respond (with respect to and up until their similarity) similarly to a given strategy. \item Note that this work around is because we have NO historical data on how coal phase out strategy affects coal plant retirement (retirements B.C. are not helpful here) \textit{therefore we have to understand coal plants holistically before we can predict their responses}
    \item There are a variety of characters a coal plant could have to make it more or less difficult to effectively apply a strategy. On top of that, the particular relationships of a these characteristics are crucial in understanding strategy efficacy. 
\end{itemize}


\subsection{What have we achieved towards solving it?}
\textcolor{red}{The main point of this paper is convincing someone we've actually done these things. BUT, we are up a creek without a paddle if we cannot motivate to them what the problem is, why we want to solve it, and why it is difficult to do so. I honestly lose track of this myself quite often because the PROBLEM of Decarbonization is so prevalent, but we need to remember what WE are doing about it.}
\begin{itemize}
    \item For the set of coal plants, we prescribe an generalized assignment of strategies that approximately minimizes the total time of un-retired coal plants (assuming fixed resources)
    \item we have given a data-driven distance from retirement which allows us to proxy efficacy of a prescribed strategy. We can then identify the easy to retire and difficult to retire subgroups 
\end{itemize}

%  Alex prescription #2 

\section{What we have}
\begin{itemize}
    \item an explanation outlining why previously retired plants shouldn't be used to predict retirement now.
    \item a dataset with a spanning view of the problem (YOU NEED TO BE ABLE TO ARGUE THIS) 
    \item A clustering of coal plants over all columns of a dataset (groups that are similar according to the 'important' data) 
    \item A labelling of planned retired plants that gives a distance from retirement 
\end{itemize}

\section{What we Need}
\begin{enumerate}
     \item \textbf{A set of retirement strategies} (also the motivation of why this set is (somewhat) fixed) 
     \item Matching (not necessarily one-to-one) of strategies onto groups (MAIN RESULT 1)
     \item A set of most problematic and least problematic coal plants according to the set of strategies above (MAIN RESULT 2) 
     \item Memorable group names/labels based on most significant characteristics
     \item A cost/resource proxy for retirement difficulty (easier) 
     \item A timeline of estimated plant retirement according to path distance + an estimated end date (from literature/simulator) 
     \item an analysis of ownership for determining "problematic" group 
     \item WE NEED A WAY TO JUSTIFY THE NUMBER OF POLICY GROUPS SELECTED (the tool gives a lot of resolutions, but we are showing how to use the tool, not what the tool is)
     \item IDEA: Once we have the set of strategies, we pick the best resolutions
\end{enumerate}

\section{what we want}
\begin{itemize}
    \item A labelling of \textit{forced/early} planned retired plants that gives a distance from retirement  
    \item A customized timeline based on resource allocation 
    \item an analysis of the problematic group 
\end{itemize}


\section{After Getting what we need}

\begin{itemize}
    \item Paper Structure 
    \item what are going after in the intro/motivation? What figures do we need to communicate our point 
    \item JEREMY/STU: How are we introducing Manifold Learning + Hyperparameter Selection + graph representations in the intro?  
    \item JEREMY/STU: methods Section
    \item Dataset Breakdown for appendix
    
\end{itemize}


% \begin{itemize}
%     \item Motivation: \textcolor{green}{FLUFF}:coal is quite bad. Global warming is bad 
    
    
%     \textcolor{blue}{Better} Current analysis of phase out structures is lack luster.
    
%     \item Given the same strategies at our disposal right now, how do we use them more effectively? We are concerned with, given the resources you have (global scale or individual NGO level) how can you best use those resources? Objective: retire coal plants. Optimization: where do we focus our resources to achieve the objective.
%     \item The difficulty here is that analyzing coal plants to match corresponding phase-out strategies is a highly dimensional problem. \textbf{why is it difficult to analyze coal plants to predict how strategies will effect their retirement}
    
% \end{itemize}

% \textbf{\textit{we have a way to predict how effective a specific strategy will be at pushing a coal plant or group of plants to retire early}}

% \subsection{ \textbf{Why this problem is hard}}
% \begin{itemize}
%     \item you must look at relationships between different facets of data in order to understand a plant's behavior. e.g. Financial data will not give you insight into a plant's financial setting/future (ESG thing) you need to understand financial data in the context of political data, etc. to fully understand a plant.
%     \item Relationships between plants are valuable not to understand plants better, but to help understand how effective policy will be at targeting multiple plants (Sid's subtle point: historical data on coal plant retirement is not helpful, but current data following the Blah blah is?? This is not immediately understandable (by Stu) so we need to motive!) 
%     \item if we really understand coal plants, we can understand why certain strategies were effective. if we understand strategy efficacy, we can predict it.
%     \item we are not suggesting strategy, but if \textit{x} strategy works on \textit{a}, we can tell you how and why it work for \textit{b} based on complex similarities
% \end{itemize}


% \subsubsection{ \textbf{what have we achieved towards solving it, what we add to the field: }
% \begin{itemize}
%     \item we are bring a way to view the problem from a wholistic perspective, something that is distinctly lacking
%     \item we are allowing comparison not only based on attributes, but based on relationships to plant we know more / less about. 
%     \item why is it crucial to understand how/why all the different aspects react with / relate to eachother. 
% \end{itemize}
% %  ╭────────────────────────────────────────────────────────────────────────────────────────╮
%  │  For future use as a template to help format tables in the paper                       |
%  ╰────────────────────────────────────────────────────────────────────────────────────────╯


%Widows & Orphans & Penalties
\widowpenalty500
\clubpenalty500
\clubpenalty=9996
\exhyphenpenalty=50 %for line-breaking at an explicit hyphen
\brokenpenalty=4991
\predisplaypenalty=10000
\postdisplaypenalty=1549
\displaywidowpenalty=1602
\floatingpenalty = 20000


% \usepackage[T1]{fontenc}
% \usepackage{fontspec}
% \setmainfont[Scale=0.85, Ligatures={Required,Common,Contextual,TeX}]{TeX Gyre Schola} % Incredible font inside latex


\begin{table*}[] \centering
%\ra{1.3}
\begin{small}
\begin{tabular}{@{}lrrrrrrrrrrrr@{}}\toprule
\textbf{Debt/Assets} & \textbf{2002} & \textbf{2003} & \textbf{2004} & \textbf{2005} & \textbf{2006} & \textbf{2007} & \textbf{2008} & \textbf{2009} & \textbf{2010} & \textbf{2011} & \textbf{2012} & \textbf{Total} \\ \midrule
\textbf{<50\%} & -\% & 10\% & 5\% & 7\% & 11\% & 2\% & -\% & -\% & -\% & -\% & 3\% & 2\%\\ \hdashline
\textbf{<50\% - 59.9\%} & 20\ & 5\ & 16\ & 7\ & 11\ & 8\ & 13\ & 11\ & 9\ & 15\ & 8\ & 11\ \\ \hdashline
\textbf{<60\% - 69.9\%} &  -\ & 10\ & 5\ & 13\ & 11\ & 10\ & 16\ & 16\ & 17\ & 11\ & 15\ & 13\ \\ \hdashline
\textbf{<70\% - 79.9\%} &  -\ & 10\ & 37\ & 13\ & 15\ & 8\ & 24\ & 29\ & 16\ & 14\ & 35\ & 22\ \\ \hdashline
\textbf{<80\% - 89.9\%} &  40\ & 10\ & 5\ & 17\ & 7\ & 24\ & 18\ & 18\ & 21\ & 6\ & 8\ & 15\ \\ \hdashline
\textbf{>90\%} &  40\ & 55\ & 32\ & 43\ & 46\ & 49\ & 29\ & 26\ & 38\ & 24\ & 35\ & 37\ \\ \midrule
\textbf{Total} &  100\% & 100\% & 100\% & 100\% & 100\% & 100\% & 100\% & 100\% & 100\% & 100\% & 100\% & 100\% \\ \midrule
\textbf{Mean} &  85\% & 80\% & 77\% & 80\% & 78\% & 85\% & 79\% & 79\% & 82\% & 77\% & 80\% & 80\% \\ \hdashline
\textbf{Median} &  85\% & 94\% & 76\% & 81\% & 80\% & 87\% & 79\% & 77\% & 82\% & 75\% & 72\% & 80\% \\ \hdashline
\textbf{No. of Projects} &  5\% & 20\% & 19\% & 30\% & 46\% & 51\% & 68\% & 38\% & 58\% & 54\% & 26\% & 415\% \\
\bottomrule
\end{tabular}
\end{small}
\caption{Initial capital structures of large projects (\$1bn.+) \emph{(Finnerty, 2013)}}
\end{table*}

\begin{table*}[] \centering
%\ra{1.3}
\begin{small}
\begin{tabular}{@{}lrrrrrrrrrr@{}}\toprule
 & \textbf{1995} & \textbf{1996} & \textbf{1997} & \textbf{1998} & \textbf{1999} & \textbf{2000} & \textbf{2001} & \textbf{2002} & \textbf{Total}\\ \midrule
\textbf{Bank Loans} & \$23.33 & \$42.83 & \$67.43 & \$56.65 & \$72.39 & \$110.89 & \$108.48 & \$62.20 & \$557.88\\ \hdashline
\textbf{Bonds} & 3.79 & 4.79 & 7.70 & 9.79 & 19.79 & 20.81 & 25.00 & 13.80 & 109.26\\ \hdashline
\textbf{Development Agencies} & 17.59 & 18.96 & 22.05 & 20.97 & 16.62 & 17.69 & 18.75 & 18.75 & 162.63\\ \hdashline
\textbf{Total Debt} & 44.71 & 66.58 & 96.98 & 87.41 & 108.80 & 149.39 & 152.23 & 94.75 & 829.77\\ \hdashline
\textbf{Equity} & 19.16 & 28.54 & 41.56 & 37.46 & 46.70 & 64.02 & 65.24 & 40.61 & 355.68\\ \midrule
\textbf{Total} & \$63.88 & \$95.12 & \$138.54 & \$124.87 & \$155.68 & \$213.40 & \$217.47 & \$135.36 & \$1185.63\\ \midrule
\textbf{Bank Loans} & 37\% & 45\% & 49\% & 45\% & 46\% & 52\% & 50\% & 46\% & 42\% \\ \hdashline
\textbf{Bonds} & 6 & 5 & 5 & 8 & 13 & 10 & 11 & 10 & 9\\ \hdashline
\textbf{Development Agencies} & 28 & 20 & 16 & 17 & 11 & 8 & 9 & 14 & 14\\ \hdashline
\textbf{Total Debt} & 70 & 70 & 70 & 70 & 70 & 70 & 70 & 70 & 70\\ \hdashline
\textbf{Equity} & 30 & 30 & 30 & 30 & 30 & 30 & 30 & 30 & 30\\ \midrule
\textbf{Total} & 100  & 100  & 100  & 100  & 100  & 100  & 100  & 100  & 100\\
\bottomrule
\end{tabular}
\end{small}
\caption{Project funding by source (US\$ billions) \emph{(Finnerty, 2013)}}
\end{table*}

\begin{table*}[] \centering
%\ra{1.3}
\begin{small}
\begin{tabular}{@{}lllrrr@{}}\toprule
\textbf{Rank} & \textbf{Lead Arranger} & \textbf{Number of Deals} & \textbf{Dollar Amount} & \textbf{Market Share} & \textbf{Equator Principles Adoption}\\ \midrule
\textbf{1} & State Bank of India & 52 & \$21,631.6 & 10.1\% & NA\\ \hdashline
\textbf{2} & Mitsubishi UFJ Financial & 88 & 9,486.1 & 4.4 & Dec 2005\\ \hdashline
\textbf{3} & Sumitomo Mitsui & 71 & 8,188.1 & 3.8 & Jan 2006\\ \hdashline
\textbf{4} & Credit Agrocole & 60 & 6,506.4 & 3.1 & Jun 2005\\ \hdashline
\textbf{5} & Mizuho Financial & 55 & 5,797.5 & 2.7 & Oct 2003\\ \hdashline
\textbf{6} & Soci\'{e}t\'{e} Generale & 55 & 5,760.5 & 2.7 & Sep 2007\\ \hdashline
\textbf{7} & BNP Paribas & 55 & 5,390.8 & 2.5 & Oct 2008\\ \hdashline
\textbf{8} & Axis Bank & 18 & 5,216.9 & 2.4 & NA\\ \hdashline
\textbf{9} & IDBI Bank & 10 & 5,162.3 & 2.4 & NA\\ \hdashline
\textbf{10} & ING & 49 & 4,916.1 & 2.3 & Jun 2003\\ \midrule
 & Others & 102 & 135,430.4 & 63.6 & \\ \midrule
 & Total Market & 615 & \$213,486.7 & 100\% & \\
\bottomrule
\end{tabular}
\end{small}
\caption{Global project bank facility lead arrangers \emph{(Finnerty, 2013)}}
\end{table*}


%  ╭──────────────────────────────────────────────────────────╮
%  │ Intro                                                    |
%  ╰──────────────────────────────────────────────────────────╯
\stuTodo{I've put in some macros for notes and todos. If my name is Sid, example usage is $\backslash$sidTodo for todos and $\backslash$sidNote for notes. (Supported for Sid, Grace, Stu and Jez.) }
\section{Introduction}

Despite numerous studies demonstrating the feasibility of phasing out coal by 2035 \textit{(cite papers here)}, real-world action remains elusive. Thus far, feasibility has not translated into practice. \textit{insert statistics about current phase out rate to prove this point} Rather than suggesting a feasible route forwards, we need to be suggesting strategies to speed up the retirement of US coal plants. We aim to highlight what's needed to retire coal plants and identify the essential policies required to achieve these ambitious goals.


\begin{figure}[h]
    % \includesvg[inkscapelatex=false,width=1\columnwidth]{svg_figs/fleetInfo/whatIf_retirement_timeline.svg}
    \includesvg[inkscapelatex=false,width=1\columnwidth]{svg_figs/fleetInfo/gen_age.svg}
    \includesvg[inkscapelatex=false,width=1\columnwidth]{svg_figs/fleetInfo/gen_ageAtRetirement.svg}
    \label{fig:1}
    \caption{\textbf{Current Coal Fleet Status - Age Breakdown} The figure illustrates the distribution of ages within the U.S. coal generator fleet, juxtaposed with their anticipated retirement ages. Notably, the average retirement age of plants that have announced planned retirement dates is 54.4 years, indicating a misalignment with imperative climate targets. Only 41.5\% (85.5 GW) of current coal capacity has announced plans to retire - with 91\% (78GW) of that capacity retiring by 2035.}
\end{figure}


Retiring coal plants in the USA poses a complicated, multi-faceted problem. Many papers (insert citations here) point to technical, financial, and environmental coal plant attributes as a way to determine low hanging fruit and prioritize retirements, utilizing a composite vulnerability score that combines these factors to guide decision-making. In addition to the technical/environmental/financial vulnerability of a coal plant, there are social, health, and political perspectives as well, just to name a few. Even with a comprehensive dataset, a composite vulnerability score approach is an oversimplification of a complex problem:

\begin{itemize}
    \item By combining data into a single score, a significant amount of detail is lost and nuances/interdependence's are rolled up into a subjective ranking. Different coal plants might operate in diverse contexts, locations, and regulatory environments. What is considered important in one scenario might not hold the same weight in another.
    \item Deciding how much weight to assign to each factor in the composite score is subjective. Different stakeholders may have different opinions on the importance of various components.
    \item A composite score lacks detail and interpretability. Does a high vulnerability score mean a coal plant is overall vulnerable, or does it excel in only one aspect while lagging in others?
\end{itemize}

While using vulnerability scores in the context of scenario modeling helps to compare the potential impacts of different phase out strategies and optimize an overall retirement plan, this approach merely produces a prescriptive list that can either be followed or deviated from - without giving insight into the strategies required to pursue coal plants. Additionally, the idea of an \textit{optimal} retirement plan does not work well within the political context of the United States where extraneous reasons can lead coal plants ready for retirement to continue operating long past their prime and/or plants in good economic standing to be shut down early. 

%  ╭────────────────────────────────────╮
%  │  no deep learning                  |
%  ╰────────────────────────────────────╯

\subsection{A Changing Retirement Landscape}

The evolution of coal plant retirements has undergone a significant transformation. 
In the past, retirements often occurred organically, primarily driven by internal financial considerations, 
changing market dynamics, or technological obsolescence \textit{insert citation}. 
These retirements were driven by the economic viability of coal plants and their ability to compete 
in the energy market. Such retirements were not directly tied to explicit environmental or climate goals. 
However, the contemporary landscape is marked by a paradigm shift in the rationale behind coal plant retirements. 
The urgency to address climate change, reduce greenhouse gas emissions, and transition toward cleaner energy 
sources has become a driving force behind retirement decisions. This shift is propelled by the recognition of 
the critical role of coal-fired power generation in contributing to global carbon emissions and exacerbating 
climate-related challenges.

While past retirements might offer insights into operational challenges and procedural aspects of phasing out 
coal plants, they do not serve as reliable indicators for predicting future retirements. Several reasons underscore 
this departure from past trends. First, the current impetus for coal phaseout arises from external factors like 
international climate agreements (e.g., the Paris Agreement), increased public awareness of environmental issues, 
and regulatory frameworks that prioritize emission reductions. These factors introduce new dynamics that were 
absent or less pronounced during the organic retirement era. Second, decisions to retire coal plants are now embedded 
in a complex web of considerations, including economic viability, technological feasibility, regulatory compliance, 
community impacts, and climate objectives. This multi-dimensional decision landscape surpasses the relatively simpler 
determinants of past retirements. The involvement of diverse stakeholders, from local communities to environmental 
organizations, has gained prominence in modern coal phaseout initiatives. Ensuring a just transition for affected 
communities and addressing social equity concerns were not as central to past retirements. Third, crypto-currency has 
also breathed life into dying coal plants. Through both PPAs and increased grid loads, server farms primarily used to 
mine bit-coin and other blockchain based currencies have began purchasing cheap power from coal plants on the brink, hungry 
for alternative or additional sources of revenue. \textit{Need citations here, heard about this through work - will need 
to get more info and make this punchier}

The shift from organic retirements to strategic retirements, motivated by climate and policy objectives, introduces complexities 
that make the application of supervised machine learning challenging in the context of predicting future coal plant retirements. 
Training on data that includes retired coal plants fails to capture the transformational shift in decision drivers. The need for 
creative retirement solutions continues to be of paramount importance - assuming coal plants will retire on their own is overly 
optimistic, neglecting the pressing need for a transition to clean energy.

%  ╭────────────────────────────────────╮
%  │  curse of dimensionality           |
%  ╰────────────────────────────────────╯

\subsection{Coal and \textit{the Curse of Dimensionality}}

The coal plant retirement problem suffers from \textit{the curse of dimensionality}, where as the dimensions of a dataset 
increase, the number of data points needed to guarantee reliable results grows exponentially. In the case of the current US coal 
fleet, we have a very limited number of coal plants with very high-dimensional data associated with each. As such, standard 
unsupervised methods fail to reliably predict which active coal plants have planned retirements on the books. Additionally, the 
current arsenal of artificial intelligence methods fall short when it comes to comparing and analyzing complex objects across different 
data disciplines. In regards to coal plants, we aim to analyze Environmental, Technical, Financial, Political, and Health data, making 
it again difficult to understand the complex interactions between all variables at a high level. 

Even advanced deep learning algorithms, 
such as Gradient Boosting, struggle to effectively overcome the curse of dimensionality and provide accurate results. When looking to 
understand the variables that are important in predicting whether a coal plant will retire, different deep learning methods give us 
different results, showing how complex the coal plant retirement problem really is (\textit{\textbf{See appendix for explanatory figs}}).

%  ╭────────────────────────────────────╮
%  │  what our approach brings          |
%  ╰────────────────────────────────────╯

\subsection{Advantages of a Novel Approach}
We aim to acknowledge the transformational nature of coal phaseout motivations and propose an analytical framework that reflects the 
dynamic and multidimensional factors influencing coal plant retirement decisions. As such, we propose a novel method of analyzing the 
US coal plant space to provide new and actionable insight. Here is what our methodology adds to the academic discipline:

\begin{enumerate}
    \item \textbf{Contextual Flexibility:} By not relying on predefined composite scores or fixed weights, our method can adapt to different scenarios, regulatory environments, and stakeholder perspectives.
    \item \textbf{Holistic Understanding:} By allowing for a more nuanced and comprehensive understanding of a coal plant's/group of coal plants' performance, across various dimensions, we enable stakeholders to make more informed decisions.
    \item \textbf{Transparency:} Our approach makes the evaluation process more transparent by presenting data and relationships directly. This transparency can build trust among stakeholders, and help to avoid the interpretability issues inherent to composite scores.
    \item \textbf{Customization:} Different coal plants have unique characteristics. Our method can accommodate these variations, providing a more accurate representation of a group's strengths and weaknesses.
\end{enumerate}

We do not claim to have solved the coal plant retirement problem, nor to have figured out the best and fastest way to retire all coal plants. 
Instead, we bring a brand new perspective that gives policy makers and other stakeholders an in-depth understanding of the coal fleet from 
multiple perspectives.


%  ╭──────────────────────────────────────────────────────────╮
%  │ Results                                                  |
%  ╰──────────────────────────────────────────────────────────╯

\section{Results}

\subsection{Distributing Plant Capacity: A Classification of the US Coal Fleet}

While the U.S. has seen a decline in coal electricity production over the past ten years, many studies indicate that coal power must be phased out entirely 
by 2030-2035 if we are to meet environmental, social, political, financial, and energy delivery impacts. A graphical representation of coal plants encodes 
multi-dimensional relationships of these factors, and when looking at distributions of these representations, we can extract relationships present in the original 
column-space of the dataset. These local relationships can provide insights for choosing and evaluating phase-out strategy effectiveness and resource allocation. 
More conventional metrics fall short in capturing the intricate relationships and nuances within this high-dimensional space.

To account for the complexity of the coal phaseout problem, we construct a graph model of the US coal fleet, using over 55 variables in the raw column space 
encompassing environmental, political, financial, and other variables. The graph and resulting groups arise from the structure of our high dimensional data set, 
whose complex relationships are distilled into a more interpretable model. Unlike traditional clustering methods, we use the graph model’s structure to build 
digestible descriptions of these groups. This allows us to illustrate which features connect the coal plants in each group, and compare group profiles across all 
available data fields.

\vspace{\baselineskip}

\highlight{How to interpret the model below:
\begin{itemize}
    \item \textit{Nodes} (circles) represent clusters of similar coal plants.
    \item \textit{Edges} (lines) connect similar clusters based on multidimensional relationships in the data.
    \item \textit{Groups} (connected nodes) make up connected components, or isolated sections of the graph.
\end{itemize}
}
%%%%%%%%%%%



%  ╭──────────────────────────────────────────────────────────╮
%  │ Coal Fleet Partition Fig                                 |
%  ╰──────────────────────────────────────────────────────────╯
\begin{figure}[H]
    \input{FIG-CoalFleetPartition}
    \caption{\textbf{Classifying the US Coal Fleet}}
    \medskip
    \footnotesize
    The resulting model has 8 unique groupings of coal plants. While our groupings are derived from every feature of the data set, looking at homogeneous data fields within each group gives a high-level and digestible overview of why certain plants are grouped together. Here, we label the groups based on some defining characteristics for increased interpretability.
    The node coloring corresponds to the percent of coal plants within each node with plans to retire. Nodes with a higher percentage of retiring coal plants are assigned a darker red. The size of each node is proportional to the total amount of carbon dioxide (CO\textsubscript{2}) emissions in 2022 from the coal plants within the node (scaled). 
    Larger nodes indicate a higher cumulative emission of CO\textsubscript{2} from the associated coal plants.
    \label{fig:coal-fleet-partition}
\end{figure}
%%%%%%%%%%%



\subsection{Understanding Tradeoffs}

\begin{figure}[H]
    \includesvg[inkscapelatex=false,width=\linewidth]{svg_figs/heatmap_small.svg}  
    \caption{\textbf{Plant Groups Comparison}}
    \medskip
    \footnotesize
    The heatmap illustrates the landscape of our coal plant groupings, delineating both the similarities and differences between groups. This helps to provide insights into potential barriers and incentives influencing retirement dynamics within each group.
    A subset of variables was chosen to aid in interpretability and readability, see SI.\ref{fig:heatmap-SIs} for full heatmap on all variables.
    \label{fig:heatmap}
\end{figure}

\subsubsection{Policy Efficacy and Optimizing Resource Allocation}



%  ╭──────────────────────────────────────────────────────────╮
%  │ Shortest Path Graphs Fig                                 |
%  ╰──────────────────────────────────────────────────────────╯
\begin{figure}[H]
    \centering
    \begin{minipage}{0.5\textwidth}
        \includesvg[inkscapelatex=false,width=\linewidth]{svg_figs/group3_path.svg}  
        \subcaption{\textbf{Group 3} - Distance to Sink Nodes}
    \end{minipage}%
    \begin{minipage}{0.5\textwidth}
        \includesvg[inkscapelatex=false,width=\linewidth]{svg_figs/group4_path.svg}  
        \subcaption{\textbf{Group 4} - Distance to Sink Nodes}
    \end{minipage}

    \begin{minipage}{0.25\textwidth}
        \includesvg[inkscapelatex=false,width=\linewidth]{svg_figs/group3_averageCashflow.svg}  
        \subcaption{Average Cashflows}
    \end{minipage}%
    \begin{minipage}{0.25\textwidth}
        \includesvg[inkscapelatex=false,width=\linewidth]{svg_figs/group3_PM2.5.svg}  
        \subcaption{PM 2.5 Emissions}
    \end{minipage}%
    \begin{minipage}{0.25\textwidth}
        \includesvg[inkscapelatex=false,width=\linewidth]{svg_figs/group4_ownership.svg}  
        \subcaption{Municipal Ownership}
    \end{minipage}%
    \begin{minipage}{0.25\textwidth}
        \includesvg[inkscapelatex=false,width=\linewidth]{svg_figs/group4_proCoal.svg}  
        \subcaption{Oppose Regulating Coal}
    \end{minipage}

    \caption{\textbf{Proximity To Retirement}}
    \medskip
    \footnotesize
    \textbf{(a, b)} Employing Dijkstra's algorithm, we determine the shortest path and corresponding distance from sink nodes to every other node in the graph. Nodes where all plants are planned for full retirement (100\% of plants within) are marked as sinks. These nodes are colored blue and labeled with an \textbf{S}. Distances signify the level of similarity; nodes with shorter paths to sinks share more similar attributes. Conversely, nodes with longer distances have more dissimilar characteristics, indicating a greater divergence between the plants contained within them and the attributes of the retiring plants within sink nodes.  This is our approximation for a \textit{coarse-grained} geodesic distance in high dimensional space. \textit{explain edge weighting here as well?}
    The edges in the graph are weighted to indicate the relationships between nodes. Nodes are sized according to the number of plants in each. 
    \textbf{(c, d, e, f)} Nodes are colored by variables that exhibit variance within the group, emphasizing inter-group distinctions. Higher values are assigned a darker red. The color scheme elucidates the spatial dynamics within connected components of the graph, revealing monotonic trends and differences in specific variables as distances from the source nodes increase.
    \label{fig:shortest-path-graphs}
\end{figure}



%%%%%%%%%%%

% Group 3:
% \begin{table}[H]
%     \centering
%     \small
%     \begin{adjustbox}{width=\textwidth}
%       \begin{tabular}{|l | l | l | l | l | l | l | l | l | l |}
%         \toprule
%         \textbf{Plant Name} & \textbf{State} & \textbf{Retirement Date} & \textbf{Ownership Type} & \textbf{Age at Retirement} & \textbf{Capacity Factor} & \textbf{Coal Nameplate (MW)} \\
%         \midrule\midrule
%         Limestone & TX & 2029 & Investor-Owned & 44 & 35\% & 1850 & \\
%         Belle River & MI & 2028 & Investor-Owned & 44 & 56\% & 1395 & \\
%         Welsh & TZ & 2028 & Investor-Owned & 51 & 44\% & 1116 & \\
%         \bottomrule
%     \end{tabular}
%   \end{adjustbox}
%   \caption{Dataset \& Data Sources Breakdown}
% \end{table}


%  ╭──────────────────────────────────────────────────────────╮
%  │ Discussion                                               |
%  ╰──────────────────────────────────────────────────────────╯


%  ╭──────────────────────────────────────────────────────────╮
%  │ Methods                                                  |
%  ╰──────────────────────────────────────────────────────────╯


%  ╭──────────────────────────────────────────────────────────╮
%  │ Post-paper                                               |
%  ╰──────────────────────────────────────────────────────────╯

\section{Supplemental Info/Figs}

%  ╭──────────────────────────────────────────────────────────╮
%  │ Variables                                                |
%  ╰──────────────────────────────────────────────────────────╯

\begin{table}[H]
    \centering
    \small
    \begin{adjustbox}{width=\textwidth}
      \begin{tabular}{|l | l | l |}
        \toprule
        \textbf{Variable} & \textbf{Source} & \textbf{Used in Graph Construction} \\
        \midrule\midrule
        EIA Plant Code & EIA eGrid Dataset - 2021 & \\
        \midrule
        Plant Name & EIA eGrid Dataset - 2021 & \\
        \midrule
        Retirement Date & Sierra Club Retired and Announced Coal & \\
        \midrule
        EIA Utility Code & EIA eGrid Dataset - 2021 & \\
        \midrule
        Retirement Status & Sierra Club Retired and Announced Coal & \textbf{X} \\
        \midrule
        Date of Last Unit or Planned Retirement & Sierra Club Retired and Announced Coal & \\
        \midrule
        Total Nameplate Capacity (MW) & EIA eGrid Dataset - 2021 & \textbf{X} \\
        \midrule
        Retiring Capacity & EIA eGrid Dataset - 2021 \& Sierra Club Retired and Announced Coal & \textbf{X} \\
        \midrule
        Average Capacity Factor & EIA eGrid Dataset - 2021 & \textbf{X} \\
        \midrule
        Generator Size Weighted Capacity Factor (\%) & EIA eGrid Dataset - 2021 & \textbf{X} \\
        \midrule
        Age & EIA eGrid Dataset - 2021 & \textbf{X} \\
        \midrule
        Generator Size Weighted Age & EIA eGrid Dataset - 2021 & \textbf{X} \\
        \midrule
        Summed Generator annual net generation (MWh) & EIA eGrid Dataset - 2021 & \textbf{X} \\
        \midrule
        Age at Retirement & Sierra Club Retired and Announced Coal & \\
        \midrule
        Percent Capacity Retiring & EIA eGrid Dataset - 2021 \& Sierra Club Retired and Announced Coal & \textbf{X} \\
        \midrule
        State & EIA eGrid Dataset - 2021 & \\
        \midrule
        2022 SO2 Emissions (tons) & EPA's Clean Air Markets Division 2021 and 2022 Annual Power Plant Emissions & \textbf{X} \\
        \midrule
        2021 vs 2022 SO2 Emissions (\%) & EPA's Clean Air Markets Division 2021 and 2022 Annual Power Plant Emissions & \textbf{X} \\
        \midrule
        2022 SO2 Emission Rate (lb/mmBtu) & EPA's Clean Air Markets Division 2021 and 2022 Annual Power Plant Emissions & \textbf{X} \\
        \midrule
        2022 NOX Emissions (tons) & EPA's Clean Air Markets Division 2021 and 2022 Annual Power Plant Emissions & \textbf{X} \\
        \midrule
        2022 NOX Emission Rate (lb/mmBtu) & EPA's Clean Air Markets Division 2021 and 2022 Annual Power Plant Emissions & \textbf{X} \\
        \midrule
        2022 CO2 Emissions (tons) & EPA's Clean Air Markets Division 2021 and 2022 Annual Power Plant Emissions & \textbf{X} \\
        \midrule
        2021 vs 2022 CO2 Emissions (\%) & EPA's Clean Air Markets Division 2021 and 2022 Annual Power Plant Emissions & \textbf{X} \\
        \midrule
        2022 CO2 Emission Rate (lb/mmBtu) & EPA's Clean Air Markets Division 2021 and 2022 Annual Power Plant Emissions & \textbf{X} \\
        \midrule
        2022 Hg Emissions (lbs) & EPA's Clean Air Markets Division 2021 and 2022 Annual Power Plant Emissions & \textbf{X} \\
        \midrule
        2022 Hg Emission Rate (lb/TBtu) & EPA's Clean Air Markets Division 2021 and 2022 Annual Power Plant Emissions & \textbf{X} \\
        \midrule
        Facility has one or more low-emitting EGUs (LEE) units that do not report hourly emissions & EPA MATS Data Consolidated - 2021 & \textbf{X} \\
        \midrule
        eGRID subregion name & EIA eGrid Dataset - 2021 & \\
        \midrule
        eGRID subregion coal generation percent (resource mix) & EIA eGrid Dataset - 2021 & \textbf{X} \\
        \midrule
        eGRID subregion gas generation percent (resource mix) & EIA eGrid Dataset - 2021 & \textbf{X} \\
        \midrule
        eGRID subregion wind generation percent (resource mix) & EIA eGrid Dataset - 2021 & \textbf{X} \\
        \midrule
        eGRID subregion solar generation percent (resource mix) & EIA eGrid Dataset - 2021 & \textbf{X} \\
        \midrule
        Hospital Admits, All Respiratory & Coal Plant Health Impacts 2019 - COBRA Pollution Transport Model Results, Provided by the Sierra Club & \textbf{X} \\
        \midrule
        Infant Mortality & Coal Plant Health Impacts 2019 - COBRA Pollution Transport Model Results, Provided by the Sierra Club & \textbf{X} \\
        \midrule
        Hospital Admits, Cardiovascular (except heart attacks) & Coal Plant Health Impacts 2019 - COBRA Pollution Transport Model Results, Provided by the Sierra Club & \textbf{X} \\
        \midrule
        \$ Work Loss Days & Coal Plant Health Impacts 2019 - COBRA Pollution Transport Model Results, Provided by the Sierra Club & \textbf{X} \\
        \midrule
        \$ Mortality (low estimate) & Coal Plant Health Impacts 2019 - COBRA Pollution Transport Model Results, Provided by the Sierra Club & \textbf{X} \\
        \midrule
        \$ Asthma Exacerbation & Coal Plant Health Impacts 2019 - COBRA Pollution Transport Model Results, Provided by the Sierra Club & \textbf{X} \\
        \midrule
        total population (ACS2018) & EPA Power Plants and Neighboring Communities & \textbf{X} \\
        \midrule
        National percentile for Demographic Index (within 3 miles of plant) & EPA Power Plants and Neighboring Communities & \textbf{X} \\
        \midrule
        National percentile for People of Color Population (within 3 miles of plant) & EPA Power Plants and Neighboring Communities & \textbf{X} \\
        \midrule
        National percentile for Low Income Population (within 3 miles of plant) & EPA Power Plants and Neighboring Communities & \textbf{X} \\
        \midrule
        National percentile for Population with Less Than High School Education (within 3 miles of plant) & EPA Power Plants and Neighboring Communities & \textbf{X} \\
        \midrule
        PM 2.5 Emissions (tons) & EPA's Clean Air Markets Division 2021 and 2022 Annual Power Plant Emissions & \textbf{X} \\
        \midrule
        PM 2.5 Emission Rate (lb/MWh) & EPA's Clean Air Markets Division 2021 and 2022 Annual Power Plant Emissions & \textbf{X} \\
        \midrule
        Average PM Results (lb/mmBtu) & EPA MATS Data Consolidated - 2021 & \textbf{X} \\
        \midrule
        Mapped Fuel Type & EPA Power Plants and Neighboring Communities & \textbf{X} \\
        \midrule
        Coal Fuels & EIA eGrid Dataset - 2021 & \textbf{X} \\
        \midrule
        Number of Coal Generators & EIA eGrid Dataset - 2021 & \textbf{X} \\
        \midrule
        Non-Coal FUELS & EIA eGrid Dataset - 2021 & \textbf{X} \\
        \midrule
        Total Cost of Emissions Control Equipment Retrofits Installed since 2012 (\$) & EIA Form 860 Dataset & \textbf{X} \\
        \midrule
        Plant Coal Percentage (\%) & EIA eGrid Dataset - 2021 & \textbf{X} \\
        \midrule
        2020 Net Cashflow & RMI Uneconomic Dispatch Dataset & \textbf{X} \\
        \midrule
        Average Cashflow (since 2012) & RMI Uneconomic Dispatch Dataset & \textbf{X} \\
        \midrule
        Forward Costs & Energy Innovation Coal Cost Crossover 3.0 Dataset & \textbf{X} \\
        \midrule
        Coal Debt Securitization Policy & RMI State Utility Policies Data & \textbf{X} \\
        \midrule
        State coal generation percent (resource mix) & EIA eGrid Dataset - 2021 & \textbf{X} \\
        \midrule
        State gas generation percent (resource mix) & EIA eGrid Dataset - 2021 & \textbf{X} \\
        \midrule
        LAT & EIA eGrid Dataset - 2021 & \textbf{X} \\
        \midrule
        LON & EIA eGrid Dataset - 2021 & \textbf{X} \\
        \midrule
        Governor Party & RMI State Utility Policies & \textbf{X} \\
        \midrule
        Legislation Majority Party & RMI State Utility Policies & \textbf{X} \\
        \midrule
        Estimated percentage who somewhat/strongly oppose setting strict limits on existing coal-fire power plants & Yale Climate Opinion Map Aggregated Dataset - 2008-2021 & \textbf{X} \\
        \midrule
        Change from 2018 to 2021 in the estimated percentage who somewhat/strongly oppose setting strict limits on existing coal-fire power plants & Yale Climate Opinion Map Aggregated Dataset - 2008-2021 & \textbf{X} \\
        \midrule
        Cheapest Renewable Resource & Energy Innovation Coal Cost Crossover 3.0 Dataset & \textbf{X} \\
        \midrule
        Difference between cheapest renewables and coal (\$/MWh) & Energy Innovation Coal Cost Crossover 3.0 Dataset & \textbf{X} \\
        \midrule
        Renewables or Coal & Energy Innovation Coal Cost Crossover 3.0 Dataset & \\
        \midrule
        Percent difference & Energy Innovation Coal Cost Crossover 3.0 Dataset & \\
        \midrule
        Sector & EIA eGrid Dataset - 2021 & \\
        \midrule
        Utility Name & EIA eGrid Dataset - 2021 & \ \\
        \midrule
        Ownership Type & RMI Coal Asset Valuation Tool \& SEPA Utility Carbon-Reduction Tracker Dataset & \textbf{X} \\
        \midrule
        Final Target Summary & SEPA Utility Carbon-Reduction Tracker & \\
        \midrule
        Attainment Year & SEPA Utility Carbon-Reduction Tracker & \\
        \midrule
        Interim Target(s) Summary & SEPA Utility Carbon-Reduction Tracker & \\
        \midrule
        Mandatory or Voluntary Target & SEPA Utility Carbon-Reduction Tracker & \\
        \midrule
        Target Scope & SEPA Utility Carbon-Reduction Tracker & \\
        \midrule
        FIPS Code & Calculated from Lat/Lon\\
        \bottomrule
      \end{tabular}
    \end{adjustbox}
    \caption{Dataset \& Data Sources Breakdown}
  \end{table}



%  ╭──────────────────────────────────────────────────────────╮
%  │ Generator data agg                                       |
%  ╰──────────────────────────────────────────────────────────╯

  \begin{table}[H]
    \centering
    \small
    \begin{adjustbox}{width=\textwidth}
      \begin{tabular}{|l | l | l |}
        \toprule
        \textbf{Variable} & \textbf{Aggregation Method} & \textbf{Comments} \\
        \midrule\midrule
        Retirement Date & First occurrence (earliest retirement date) & 11.1\% of plants (23.2\% of plants with retirement plans) only have partial retirement plans, reflected in the row below \\
        \midrule
        Date of Last Unit or Planned Retirement & Maximum value (latest retirement date) & \\
        \midrule
        Retirement Status & Maximum value & 0: No Retirement Plans, 1: Planned Partial Retirement (e.x. 2 of 3 coal generators retiring), 2: Full Retirement Planned \\
        \midrule
        Total Nameplate Capacity (MW) & Sum of nameplate capacities of all coal generators at plant & \\
        \midrule
        Retiring Capacity & Sum of generator retiring capacities (MW) & \\
        \midrule
        Capacity Factor & Mean (average) capacity factor across generators & \\
        \midrule
        Generator Size Weighted Capacity Factor (\%)  & Capacity-weighted average capacity factor & \\
        \midrule
        Age & Mean (average) age & \\
        \midrule
        Generator Size Weighted Age & Capacity-weighted average age & \\
        \midrule
        Summed Generator annual net generation (MWh) & Sum of generator annual net generation & \\
        \midrule
        Age at Retirement & Maximum age at retirement (oldest unit retiring) & \\
        \midrule
        Percent Capacity Retiring & Percentage of plant coal capacity retiring & $\textnormal{Sum of retiring coal generators at plant (MW)} \div \textnormal{Sum of all coal generators at plant (MW)}$\\

        \bottomrule
    \end{tabular}
    \end{adjustbox}
    \caption{Aggregation of Generator-Level Variables for Plant-Level Dataset}
\end{table}



%  ╭──────────────────────────────────────────────────────────╮
%  │ Impute data                                              |
%  ╰──────────────────────────────────────────────────────────╯

\begin{table}[H]
  \centering
  \small
  \begin{adjustbox}{width=\textwidth}
    \begin{tabular}{|l | r | r |}
      \hline
      \textbf{Variable} &  \textbf{Plants Missing Data (count)} &  \textbf{Capacity (MW) Missing (\%)} \\
      \hline\hline
      2022 SO2 Emissions (tons)                          &              21 &            4.77\% \\ \hline
      2021 vs 2022 SO2 Emissions (\%)                     &              21 &            4.77\% \\ \hline
      2022 SO2 Emission Rate (lb/mmBtu)                  &              21 &            4.77\% \\ \hline
      2022 NOX Emissions (tons)                          &              21 &            4.77\% \\ \hline
      2022 NOX Emission Rate (lb/mmBtu)                  &              21 &            4.77\% \\ \hline
      2022 CO2 Emissions (tons)                          &              22 &            4.82\% \\ \hline
      2021 vs 2022 CO2 Emissions (\%)                     &              22 &            4.82\% \\ \hline
      2022 CO2 Emission Rate (lb/mmBtu)                  &              22 &            4.82\% \\ \hline
      2022 Hg Emissions (lbs)                            &              52 &            11.49\% \\ \hline
      2022 Hg Emission Rate (lb/TBtu)                    &              52 &            11.49\% \\ \hline
      Hospital Admits, All Respiratory                   &              16 &            2.23\% \\ \hline
      Infant Mortality                                   &              16 &            2.23\% \\ \hline
      Hospital Admits, Cardiovascular                     &              16 &            2.23\% \\ \hline
      \$ Work Loss Days                                   &              16 &            2.23\% \\ \hline
      \$ Mortality (low estimate)                         &              16 &            2.23\% \\ \hline
      \$ Asthma Exacerbation                              &              16 &            2.23\% \\ \hline
      PM 2.5 Emssions (tons)                             &               1 &            0.46\% \\ \hline
      PM 2.5 Emission Rate (lb/MWh)                      &               1 &            0.02\% \\ \hline
      Average PM Results (lb/mmBtu)                      &              45 &            15.57\% \\ \hline
      2020 Net Cashflow                                  &              28 &            5.69\% \\ \hline
      Average Cashflow                                   &              28 &            5.69\% \\ \hline
      Forward Costs                                      &              15 &            2.46\% \\ \hline
      Cheapest Renewable Resource                        &              15 &            2.46\% \\ \hline
      Difference between cheapest renewables and coal... &              15 &            2.46\% \\ \hline
      Ownership Type                                     &              52 &            25.08\% \\ \hline
      \end{tabular}
  \end{adjustbox}
  \caption{Random-Sample Imputed Values Breakdown}
\end{table}
\begin{figure}[!ht]
    \centering
    \includesvg[inkscapelatex=false,width=\linewidth]{figs/svgs/ml_featureImportance_accuracy.svg}  
    \caption{\textbf{Variable Importances in Predicting Planned Retirements}}
    \medskip
    \footnotesize
    \highlight{
        Different machine learning methods identify distinct features as highly influential. These divergent results emphasize the challenge of identifying consistent influential features across models, indicative of complexities in our dataset and motivating the need for novel analytical techniques.
    }

    \smallskip  % Added line skip here

    Each bar represents the importance of a feature in predicting the target variable. Features with an importance of 1 are deemed to provide significant information and contribute substantially to explaining the variance in the target variable. When multiple features have high importances (such as \textit{Total Nameplate Capacity (MW), Age} and \textit{2022 Hg Emission Rate} do in Gradient Boosting), their cumulative effect on the model's predictions is substantial; the model relies heavily on the information provided by these features.
    The many other moderately important features (importances 0.1-0.5) contribute, but to a lesser extent. The hierarchy of importance highlights the dominance of the feature with importance 1. 

    \label{fig:ML-featureImportance}
\end{figure}


\begin{figure}[h!]
    \centering
    \includesvg[inkscapelatex=false,width=\linewidth]{figs/svgs/ml_prediction_accuracy.svg}  
    \caption{\textbf{Predicting Planned Retirements} with Supervised Learning}
    \medskip
    \footnotesize
    \highlight{
        Most notably, none of these models can accurately predict planned retirement in our coal plant dataset (averaging an F1-Score above 80\%). The size of our dataset (and size of the coal fleet), as well as the inherent unpredictability of external influences and the dynamic nature of the energy sector, most likely contribute to the observed model performance. 
    }

    \smallskip  % Added line skip here
    
    We use multiple supervised machine learning methods (Deciscion Tree, Elastic Net, Gradient Boosting, Lasso Regresison, Random Forest) and evaluation metrics (Train-Test Split, Cross Validation) to predict coal plant retirements and evaluate prediction accuracy.
    An F1-Score (as seen on the \textit{y axis}) is a metric that balances precision and recall, providing a single value that indicates a model's overall accuracy by considering both false positives and false negatives; higher F1 scores (ranging from 0 to 1) indicate better performance.
    While a Train-Test Split produces  better results, we see the evaluation is quite sensitive to how the dataset is split, especially because our coal dataset is small. Additionally, Train-Test Split uses a portion of the data only for training and another portion only for testing, so is prone to overfitting.
    Cross Validation provides a more robust estimate of model performance by using multiple splits, as well as utilizing the entire dataset for both training and testing. However, this evaluation metric is significaltly more computationally intensive than a Train-Test Split, typically ued for thorough evaluations of ML results. We have shown both methods here as they each have their advantages and drawbacks, and both represent standard methods for evaluating supervised machine learning results.
    \label{fig:ML-Predictions}
\end{figure}


\begin{figure}[htb]
    \centering
    \includesvg[inkscapelatex=false,width=\linewidth]{svg_figs/heatmap_large.svg}  
    \caption{\textbf{Full Plant Groups Comparison Heatmap}}
    \medskip
    \footnotesize

    \smallskip  
    
    % Insert Fig Caption Here

    \label{fig:heatmap-SIs}
\end{figure}

\begin{figure}[htb]
    \centering
    \includesvg[inkscapelatex=false,width=1\columnwidth]{svg_figs/fleetInfo/gen_age.svg}
    \includesvg[inkscapelatex=false,width=1\columnwidth]{svg_figs/fleetInfo/gen_ageAtRetirement.svg}
    \caption{\textbf{Coal Generator Age Comparison}}
    \medskip
    \footnotesize

    \smallskip  
    
    % Insert Fig Caption Here

    \label{fig:gen-ages}
\end{figure}

\end{document}








%  ╭──────────────────────────────────────────────────────────╮
%  │ EXTRAs - will be moved/deleted                           |
%  ╰──────────────────────────────────────────────────────────╯


% \begin{figure}[H]
%     \includesvg[inkscapelatex=false,width=1\columnwidth]{svg_figs/group_map.svg}
%     \label{fig:2}
%     \caption{\textbf{Plant Group Locations} Illustrates the locations of the plants in every group.}
% \end{figure}



% \begin{figure}[H]
%     \includesvg[inkscapelatex=false,width=1\columnwidth]{svg_figs/ml_featureImportance_accuracy.svg}
%     \includesvg[inkscapelatex=false,width=1\columnwidth]{svg_figs/ml_prediction_accuracy.svg}
%     \label{fig:2}
%     \caption{\textbf{Predicting Planned Retirements in the US Coal Plant Fleet} In the top figure, there is a clear difference between each method, with no clear consensus on a subset of variables that are important for predicting planned retirement. In the bottom figure, F1-Scores averaging 0.5 indicate that across 100 different runs, these models do not preform well in terms of their ability to balance precision and recall. 0.8 is considered the standard threshold at which F1-Scores indicate a model has preformed well. In other words, these methods poorly predict planned retirement and the key variables responsible for it.}
% \end{figure}



% \begin{figure}[H]

%     \begin{minipage}{0.5\textwidth}
%         \includegraphics[width=\linewidth]{png_figs/shortest_path/group_1_vulnerability.png}
%         \subcaption{Group 1 Subgraph}
%     \end{minipage}%
%     \begin{minipage}{0.5\textwidth}
%         \includegraphics[width=\linewidth]{png_figs/shortest_path/group_2_vulnerability.png}
%         \subcaption{Group 2 Subgraph}
%     \end{minipage}
    
%     \begin{minipage}{0.5\textwidth}
%         \includegraphics[width=\linewidth]{png_figs/shortest_path/group_3_vulnerability.png}
%         \subcaption{Group 3 Subgraph}
%     \end{minipage}%
%     \begin{minipage}{0.4\textwidth}
%         \raggedleft
%         \includesvg[inkscapelatex=false,width=1\linewidth]{svg_figs/vulnerability_barPolar.svg}
%         \subcaption{Vulnerability Proxies per Group (by coal plant)}
%     \end{minipage}

%     \caption{\textbf{Proximity To Retirement Graphs} Nodes representing coal plants with the highest percentage of retiring capacity per plant are identified as target nodes. The edges in the graph are weighted to indicate the relationships between nodes. Employing Dijkstra's algorithm, we determine the shortest path and corresponding distance from these target nodes to every other node in the graph. This distance signifies the level of similarity; nodes with shorter paths to target nodes share more similar attributes with the retiring nodes. Conversely, nodes with longer distances have more dissimilar characteristics, indicating a greater divergence from the attributes of the retiring plants within nodes. \textit{Explain edge weighting here as well?}}
%     \label{fig:3}
% \end{figure}

% \begin{figure}[H]

%     \begin{minipage}{0.5\textwidth}
%         \includegraphics[width=\linewidth]{png_figs/shortest_path/group_1_vulnerability.png}  
%     \end{minipage}%
%     \begin{minipage}{0.5\textwidth}
%         \includegraphics[width=\linewidth]{png_figs/shortest_path/group_2_vulnerability.png}  
%     \end{minipage}
%     \begin{minipage}{0.5\textwidth}
%         \includegraphics[width=\linewidth]{png_figs/shortest_path/group_3_vulnerability.png}  
%     \end{minipage}%
%     \begin{minipage}{0.4\textwidth}
%         \raggedleft
%         \includesvg[inkscapelatex=false,width=1\linewidth]{svg_figs/vulnerability_barPolar.svg}
%     \end{minipage}
        
%     \caption{\textbf{Proximity To Retirement Graphs} Nodes representing coal plants with the highest percentage of retiring capacity per plant are identified as target nodes. The edges in the graph are weighted to indicate the relationships between nodes. Employing Dijkstra's algorithm, we determine the shortest path and corresponding distance from these target nodes to every other node in the graph.
%     % This is our approximation for a "coarse-grained" geodesic distance in high dimensional space
%     This distance signifies the level of similarity; nodes with shorter paths to target nodes share more similar attributes with the retiring nodes. Conversely, nodes with longer distances have more dissimilar characteristics, indicating a greater divergence between the plants contained within them and the attributes of the retiring plants within target nodes. \textit{explain edge weighting here as well?}}
%     \label{fig:3}
% \end{figure}

% \begin{figure}[H]
%     \centering
%     \includegraphics[width=1\textwidth]{png_figs/shortest_path/group_1_vulnerability.png}  
%     \includegraphics[width=1\textwidth]{png_figs/shortest_path/group_2_vulnerability.png}  
%     \includegraphics[width=1\textwidth]{png_figs/shortest_path/group_3_vulnerability.png}  
%     %     \begin{tikzpicture}[overlay, remember picture, on background layer]
%     %     % Legend elements
%     %     \foreach \i in {0,1,2,3,4} {
%     %         \node [rectangle, fill=col\i, minimum size=0.3cm] at ({(\i+0.7)*2.5-7.5+0.3}, 0.5) {};
%     %         \node [right] at ({(\i+0.7)*2.5-7.2+0.3}, 0.5) {Group \i};}
%     %     \foreach \i in {5, 6, 7, 8} {
%     %         \node [rectangle, fill=col\i, minimum size=0.3cm] at ({(\i-3.8)*2.5-7.5+0.3}, -0.1) {};
%     %         \node [right] at ({(\i-3.8)*2.5-7.2+0.3}, -0.1) {Group \i};}
%     % \end{tikzpicture}
%     \caption{\textbf{Coal Fleet Graph} Each group has a unique mix of plant attributes across all categories. \textit{will need to find an additional model that highlights another interesting facet of the coal fleet}}
%     \label{fig:3}
% \end{figure}

% \begin{figure}
%     \includesvg[inkscapelatex=false,width=1\columnwidth]{svg_figs/correlation_matrix.svg}
%     \label{fig:s1}
%     \caption{\textbf{Variable Correlation Matrix} Note how the ret\_STATUS variable has a distinct lack of correlation to other variables}
% \end{figure}


