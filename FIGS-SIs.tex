\begin{figure}[htb]
    \centering
    \includesvg[inkscapelatex=false,width=1\columnwidth]{svg_figs/fleetInfo/gen_age.svg}
    \includesvg[inkscapelatex=false,width=1\columnwidth]{svg_figs/fleetInfo/gen_ageAtRetirement.svg}
    \caption{\textbf{Coal Generator Age Comparison}}
    \medskip
    \footnotesize

    \smallskip  
    
    % Insert Fig Caption Here

    \label{fig:gen-ages}
\end{figure}

\begin{figure}[htb]
    \centering
    \includesvg[inkscapelatex=false,width=\linewidth]{svg_figs/heatmap_large.svg}  
    \caption{\textbf{Full Plant Groups Comparison Heatmap}}
    \medskip
    \footnotesize

    \smallskip  
    
    Note that the unclustered group has large standard deviations in most columns as they are \textit{not} similar... This has been displayed moreso as a reminder than 3.14\% of coal 
    capacity remains outside the scope of this analysis.

    \label{fig:heatmap-SIs}
\end{figure}

\begin{figure}[htb]
    \centering

    \begin{minipage}{0.95\textwidth}
        \includesvg[inkscapelatex=false,width=1\columnwidth]{svg_figs/distribution-cracking.svg}
        \subcaption{Cracking Distribution of Capacity (minimizing variance across groups)}
    \end{minipage}

    \medskip

    \begin{minipage}{0.95\textwidth}
        \includesvg[inkscapelatex=false,width=1\columnwidth]{svg_figs/distribution-minimized.svg}
        \subcaption{Minimizing the Largest Share of Capacity (minimizing the maximum value of a group)}
    \end{minipage}%
    
    \caption{\textbf{Model Distributions} - Used for Model Selection}
    \medskip
    \footnotesize

    \smallskip  
    
    The \textit{x axis} displays the number of connected components (groups) in each graph, and the \textit{y axis} displays the variance (specified in by the chart caption) of Total Nameplate Capacity (MW) across connected components (groups).
    \textbf{(a)} The goal is to pick the number of connected components (groups) in which the variance of capacity between groups reaches an inflection point (the elbow of the graph).
    \textbf{(b)} The goal is to select a model that minimizes the maximum capacity contained within one group (we don't want a single group of plants to contain over 50\% of total US coal capacity).

    \label{fig:run-measures}
\end{figure}



%  ╭──────────────────────────────────────────────────────────╮
%  │ Shortest Path Graphs Fig                                 |
%  ╰──────────────────────────────────────────────────────────╯
\begin{figure}[htb]
    \centering
    \begin{minipage}{0.5\textwidth}
        \includesvg[inkscapelatex=false,width=\linewidth]{svg_figs/group3_path.svg}  
        \subcaption{\textbf{Group 3} - Distance to Sink Nodes}
    \end{minipage}%
    \begin{minipage}{0.5\textwidth}
        \includesvg[inkscapelatex=false,width=\linewidth]{svg_figs/group4_path.svg}  
        \subcaption{\textbf{Group 4} - Distance to Sink Nodes}
    \end{minipage}

    \begin{minipage}{0.25\textwidth}
        \includesvg[inkscapelatex=false,width=\linewidth]{svg_figs/group3_averageCashflow.svg}  
        \subcaption{Average Cashflows}
    \end{minipage}%
    \begin{minipage}{0.25\textwidth}
        \includesvg[inkscapelatex=false,width=\linewidth]{svg_figs/group3_PM2.5.svg}  
        \subcaption{PM 2.5 Emissions}
    \end{minipage}%
    \begin{minipage}{0.25\textwidth}
        \includesvg[inkscapelatex=false,width=\linewidth]{svg_figs/group4_ownership.svg}  
        \subcaption{Municipal Ownership}
    \end{minipage}%
    \begin{minipage}{0.25\textwidth}
        \includesvg[inkscapelatex=false,width=\linewidth]{svg_figs/group4_proCoal.svg}  
        \subcaption{Oppose Regulating Coal}
    \end{minipage}

    \caption{\textbf{Proximity To Retirement} - Supporting Sub-Graph Analysis}
    \medskip
    \footnotesize
    \textbf{(a)} Employing Dijkstra's algorithm, we determine the shortest path and corresponding distance from sink nodes to every other node in the graph. Nodes where all plants are planned for full retirement (100\% of plants within) are marked as sinks. These nodes are colored blue and labeled with an \textbf{S}. Distances signify the level of similarity; nodes with shorter paths to sinks share more similar attributes. Conversely, nodes with longer distances have more dissimilar characteristics, indicating a greater divergence between the plants contained within them and the attributes of the retiring plants within sink nodes.  This is our approximation for a \textit{coarse-grained} geodesic distance in high dimensional space. \textit{explain edge weighting here as well?}
    The edges in the graph are weighted to indicate the relationships between nodes. Nodes are sized according to the number of plants in each. 
    \textbf{(b, c)} Nodes are colored by variables that exhibit variance within the group, emphasizing inter-group distinctions. Higher values are assigned a darker red. The color scheme elucidates the spatial dynamics within connected components of the graph, revealing monotonic trends and differences in specific variables as distances from the source nodes increase.
    \label{fig:group3-shortestPath-graphs}
\end{figure}