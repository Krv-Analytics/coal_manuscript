%  ╭─────────────────────────────────────────────────────────────────────────────────╮
%  │  no deep learning - changing retirement landscape                               |
%  ╰─────────────────────────────────────────────────────────────────────────────────╯

\subsection{A Changing Retirement Landscape}\label{subsec:changing-retirement-landscape}

\stuNote{On condensing these three subsections - I think a lot of this info speaks to \textit{Why this problem is hard} which needs to stated as soon as possible (like right after saying what we did and why it is important - before introduction of method)}

The evolution of coal plant retirements has undergone a significant transformation. 
In the past, retirements often occurred organically, primarily driven by internal financial considerations, 
changing market dynamics, or technological obsolescence \textit{insert citation}. 
These retirements were driven by the economic viability of coal plants and their ability to compete 
in the energy market. Such retirements were not directly tied to explicit environmental or climate goals. 
However, the contemporary landscape is marked by a paradigm shift in the rationale behind coal plant retirements. 
The urgency to address climate change, reduce greenhouse gas emissions, and transition toward cleaner energy 
sources has become a driving force behind retirement decisions. This shift is propelled by the recognition of 
the critical role of coal-fired power generation in contributing to global carbon emissions and exacerbating 
climate-related challenges.

While past retirements might offer insights into operational challenges and procedural aspects of phasing out 
coal plants, they do not serve as reliable indicators for predicting future retirements. Several reasons underscore 
this departure from past trends. First, the current impetus for coal phaseout arises from external factors like 
international climate agreements (e.g., the Paris Agreement), increased public awareness of environmental issues, 
and regulatory frameworks that prioritize emission reductions. These factors introduce new dynamics that were 
absent or less pronounced during the organic retirement era. Second, decisions to retire coal plants are now embedded 
in a complex web of considerations, including economic viability, technological feasibility, regulatory compliance, 
community impacts, and climate objectives. This multi-dimensional decision landscape surpasses the relatively simpler 
determinants of past retirements. The involvement of diverse stakeholders, from local communities to environmental 
organizations, has gained prominence in modern coal phaseout initiatives. Ensuring a just transition for affected 
communities and addressing social equity concerns were not as central to past retirements. Third, crypto-currency has 
also breathed life into dying coal plants. Through both PPAs and increased grid loads, server farms primarily used to 
mine bit-coin and other blockchain based currencies have began purchasing cheap power from coal plants on the brink, hungry 
for alternative or additional sources of revenue. \textit{Need citations here, heard about this through work - will need 
to get more info and make this punchier}

The shift from organic retirements to strategic retirements, motivated by climate and policy objectives as well as declining renewable
energy prices and government subsidies, introduces complexities 
that make the application of supervised machine learning challenging in the context of predicting future coal plant retirements. 
Training on data that includes retired coal plants fails to capture the transformational shift in decision drivers. The need for 
creative retirement solutions continues to be of paramount importance - assuming coal plants will retire on their own is overly 
optimistic, neglecting the pressing need for a transition to clean energy.