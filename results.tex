\section{Results}

While the U.S. has seen a decline in coal electricity production over the past ten years, many studies indicate that coal power must be phased out entirely 
by 2030-2035 if we are to meet environmental, social, political, financial, and energy delivery impacts. A graphical representation of coal plants encodes 
multi-dimensional relationships of these factors, and when looking at distributions of these representations, we can extract relationships present in the original 
column-space of the dataset. These local relationships can provide insights for choosing and evaluating phase-out strategy effectiveness and resource allocation. 
More conventional metrics fall short in capturing the intricate relationships and nuances within this high-dimensional space.

To account for the complexity of the coal phaseout problem, we construct a graph model of the US coal fleet, using over 70 variables in the raw column space 
encompassing environmental, political, financial, and other variables. The graph and resulting groups arise from the structure of our high dimensional data set, 
whose complex relationships are distilled into a more interpretable model. Unlike traditional clustering methods, we use the graph model’s structure to build 
digestible descriptions of these groups. This allows us to illustrate which features connect the coal plants in each group, and compare group profiles across all 
available data fields.

\vspace{\baselineskip}

\highlight{How to interpret the model below:
\begin{itemize}
    \item \textit{Nodes} (circles) represent clusters of similar coal plants.
    \item \textit{Edges} (lines) connect similar clusters based on multidimensional relationships in the data.
    \item \textit{Groups} (connected nodes) make up connected components, or isolated sections of the graph.
\end{itemize}
}

%%%%%%
% May need to add this into another doc to position properly
%%%%%

\begin{figure}[H]
    \input{FIG-CoalFleetPartition}
    \caption{\textbf{Classifying the US Coal Fleet} The resulting model has 8 unique groupings of coal plants. 
    While our groupings are derived from every feature of the data set, looking at homogeneous data fields within 
    each group gives a high-level and digestible overview of why certain plants are grouped together. Here, we label 
    the groups based on some defining characteristics for increased interpretability.
    The node coloring corresponds to the percent of coal plants within each node with plans to retire. 
    Nodes with a higher percentage of retiring coal plants are assigned a darker red. The size of each node is 
    proportional to the total amount of carbon dioxide (CO\textsubscript{2}) emissions in 2022 from the coal plants within the node (scaled). 
    Larger nodes indicate a higher cumulative emission of CO\textsubscript{2} from the associated coal plants.}
    \label{fig:coal-fleet-partition}
\end{figure}

\subsection{Understanding Tradeoffs}

\subsubsection{Policy Efficacy and Optimizing Resource Allocation}