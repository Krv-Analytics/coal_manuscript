\section{Results}

\subsection{Distributing Plant Capacity: A Classification of the US Coal Fleet}

While the U.S. has seen a decline in coal electricity production over the past ten years, many studies indicate that coal power must be phased out entirely 
by 2030-2035 if we are to meet environmental, social, political, financial, and energy delivery impacts. A graphical representation of coal plants encodes 
multi-dimensional relationships of these factors, and when looking at distributions of these representations, we can extract relationships present in the original 
column-space of the dataset. These local relationships can provide insights for choosing and evaluating phase-out strategy effectiveness and resource allocation. 
More conventional metrics fall short in capturing the intricate relationships and nuances within this high-dimensional space.

To account for the complexity of the coal phaseout problem, we construct a graph model of the US coal fleet, using over 55 variables in the raw column space 
encompassing environmental, political, financial, and other variables. The graph and resulting groups arise from the structure of our high dimensional data set, 
whose complex relationships are distilled into a more interpretable model. Unlike traditional clustering methods, we use the graph model’s structure to build 
digestible descriptions of these groups. This allows us to illustrate which features connect the coal plants in each group, and compare group profiles across all 
available data fields.

\vspace{\baselineskip}
%%%%%%%%%%%



%  ╭──────────────────────────────────────────────────────────╮
%  │ Coal Fleet Partition Fig                                 |
%  ╰──────────────────────────────────────────────────────────╯
\begin{figure}[H]
    \input{FIG-CoalFleetPartition}
    \caption{\textbf{Classifying the US Coal Fleet}}
    \medskip
    \footnotesize
    The resulting model has 8 unique groupings of coal plants. While our groupings are derived from every feature of the data set, looking at homogeneous data fields within each group gives a high-level and digestible overview of why certain plants are grouped together. Here, we label the groups based on some defining characteristics for increased interpretability.
    The node coloring corresponds to the percent of coal plants within each node with plans to retire. Nodes with a higher percentage of retiring coal plants are assigned a darker red. The size of each node is proportional to the total amount of carbon dioxide (CO\textsubscript{2}) emissions in 2022 from the coal plants within the node (scaled). 
    Larger nodes indicate a higher cumulative emission of CO\textsubscript{2} from the associated coal plants.
    \label{fig:coal-fleet-partition}
\end{figure}
%%%%%%%%%%%



\subsection{Understanding Tradeoffs}

\begin{figure}[H]
    \includesvg[inkscapelatex=false,width=\linewidth]{svg_figs/heatmap_small.svg}  
    \caption{\textbf{Plant Groups Comparison}}
    \medskip
    \footnotesize
    The heatmap illustrates the landscape of our coal plant groupings, delineating both the similarities and differences between groups. This helps to provide insights into potential barriers and incentives influencing retirement dynamics within each group.
    A subset of variables was chosen to aid in interpretability and readability, see SI.\ref{fig:heatmap-SIs} for full heatmap on all variables.
    \label{fig:heatmap}
\end{figure}

\subsubsection{Policy Efficacy and Optimizing Resource Allocation}

%  ╭──────────────────────────────────────────────────────────╮
%  │ Group 3 - why sub-graph analysis?                        |
%  ╰──────────────────────────────────────────────────────────╯
\begin{figure}[H]
    \centering
    \highlight{How to interpret the models below:
        \begin{itemize}
            \item \textit{Nodes} (circles) represent clusters of similar coal plants, sized here by the number of plants within.
            \item \textit{Edges} (lines) connect similar nodes based on multidimensional relationships in the data. 
            \item \textit{Groups} (connected nodes) make up connected components, or isolated sections of the graph.
        \end{itemize}
    }

    \bigskip

    \begin{minipage}{0.32\textwidth}
        \includesvg[inkscapelatex=false,width=\linewidth]{svg_figs/group3_percCapMax.svg}  
        \subcaption{Node colors represent the maximum percentage of retiring coal capacity at a plant within the node.}
    \end{minipage}%
    \hfill
    \begin{minipage}{0.32\textwidth}
        \includesvg[inkscapelatex=false,width=\linewidth]{svg_figs/group3-percCapAve.svg}  
        \subcaption{Colored by the average percentage of coal capacity retiring within each node.}
    \end{minipage}
    \hfill
    \begin{minipage}{0.32\textwidth}
        \includesvg[inkscapelatex=false,width=\linewidth]{svg_figs/group3-subGraphs.svg}  
        \subcaption{Subgraph structures identified via label propagation community detection, colored by sub-structure.}
    \end{minipage}

    \medskip

    \begin{minipage}{1\textwidth}
        \begin{adjustbox}{width=\textwidth}
            \begin{small}
                \begin{tabular}{cllllllll}
                    \multicolumn{1}{l}{\textbf{Sub-Graph Number}} & Plant Name & Retirement Date & State & Ownership Type & Sector & Primary driver & Secondary driver \\
                    \midrule
                    \multirow{7}{*}{0} & Cumberland (TN) & 2026 & TN & Federal & Electric Utility & Air Quality Compliance - SO\textsubscript{2} \& NO\textsubscript{x} & Natural Gas Conversion / Not cost competitive  \\
                    & Rockport & 2028 & IN & Investor Owned & Electric Utility & Air Quality Compliance - SO\textsubscript{2} & Financial - Install Pollution Controls or Shut Down \\
                    & White Bluff & 2028 & AR & Investor Owned & Electric Utility & Clean Air Act Violation & \\
                    & Belle River & 2028 & MI & Investor-Owned & Electric Utility & Legal Challenges over Energy Pricing & Public Health Concerns - High Pollution \\
                    & Welsh & 2028 & TX & Investor Owned & Electric Utility & Air Quality Compliance - SO\textsubscript{2} \& NO\textsubscript{x} & Financial - Install Pollution Controls or Shut Down \\
                    & Limestone & 2029 & TX & Investor Owned & IPP Non-CHP & Cost Competition with Renewables + Utility Climate Targets & Air Quality Compliance - SO\textsubscript{2} \& NO\textsubscript{x} \\
                    & Independence Steam Electric Station & 2030 & AR & Investor Owned & Electric Utility & Clean Air Act Violation & Costly Compliance with Regional Haze Program \\
                    \midrule
                    \multirow{4}{*}{1} & Coleto Creek & 2027 & TX & Investor Owned & IPP Non-CHP & Coal Waste Compliance - Too Expensive to Comply & Not cost competitive \\
                    & J K Spruce & 2027.0 & TX & Municipal & Electric Utility & City Carbon Neutrality Target & Air Quality Compliance - NO\textsubscript{x} \& Ozone \\
                    & J T Deely & 2024.0 & TX & Municipal & Electric Utility & Pollution Control Updates - Too Expensive to Comply &  \\
                    & Tolk & 2037.0 & TX & Investor Owned &  \\
                    & Belle River & 2028 & MI & Investor-Owned & Electric Utility & Legal Challenges over Energy Pricing & Public Health Concerns - High Pollution \\
                    \midrule
                    \multirow{1}{*}{2} & Miami Fort & 2027 & OH & Investor Owned & IPP Non-CHP & Pollution Control Updates - Too Expensive to Comply & Net-Zero Emissions Target (Vistra) \\
                    \midrule
                \end{tabular}
            \end{small}
        \end{adjustbox}
        \subcaption{Sub-Graph Retirement Drivers}
    \end{minipage}    

    \medskip

    \begin{minipage}{1\textwidth}
        \includesvg[inkscapelatex=false,width=\linewidth]{svg_figs/group3-heatmap.svg}  
        \subcaption{Sub-Graph Retirement Archetype Classification (all plants with full planned retirements in (c) sub-graphs)}
    \end{minipage}


    \caption{\textbf{High Health Impact Plants (Group 3) Subgraph Analysis}}
    \medskip
    \footnotesize
    \textbf{(a)} Nearly every node contains at least one plant with plans to fully retire. In such a case (as opposed to when a higher portion of nodes contain \textit{only} non-retired plants),
    we opt to use an algorithm designed for community detection in network analysis to break up our component and understand the different retirement archetypes within
    our group of High Health Impact Plants. See SI.\ref{fig:group3-shortestPath-graphs} for a more detailed view of subgraph structures and proximity to fully retired nodes.
    \textbf{(b)} While the majority of nodes in (a) contain plants that will be fully retired, we can see a distribution of retirements across our nodes. This indicates that our hotspot of retirement plants (the cluster of red nodes at the middle-left of our group) 
    are representative of only a certain classification of coal plant with high health impacts.
    \textbf{(c)} Here we find three subgraph structures within our group, indicating that 
    there are three different retirement archetypes by which we can classify our High Health Impact Plants.
    \textbf{(d)} Retirement Archetype Comparison, comparing only plants with full planned retirement in each sub-group. See SI.\ref{group3-subgraph-Table} for a breakdown of the plants in each subgroup and which ones have full planned retirements.
\end{figure}


%%%%%%%%%%%

% Group 3:
% \begin{table}[H]
%     \centering
%     \small
%     \begin{adjustbox}{width=\textwidth}
%       \begin{tabular}{|l | l | l | l | l | l | l | l | l | l |}
%         \toprule
%         \textbf{Plant Name} & \textbf{State} & \textbf{Retirement Date} & \textbf{Ownership Type} & \textbf{Age at Retirement} & \textbf{Capacity Factor} & \textbf{Coal Nameplate (MW)} \\
%         \midrule\midrule
%         Limestone & TX & 2029 & Investor-Owned & 44 & 35\% & 1850 & \\
%         Belle River & MI & 2028 & Investor-Owned & 44 & 56\% & 1395 & \\
%         Welsh & TZ & 2028 & Investor-Owned & 51 & 44\% & 1116 & \\
%         \bottomrule
%     \end{tabular}
%   \end{adjustbox}
%   \caption{Dataset \& Data Sources Breakdown}
% \end{table}