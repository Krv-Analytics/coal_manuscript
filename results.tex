\section{Results}

\subsection{Distributing Plant Capacity: A Classification of the US Coal Fleet}

While the U.S. has seen a decline in coal electricity production over the past ten years, many studies indicate that coal power must be phased out entirely 
by 2030-2035 if we are to meet environmental, social, political, financial, and energy delivery impacts. A graphical representation of coal plants encodes 
multi-dimensional relationships of these factors, and when looking at distributions of these representations, we can extract relationships present in the original 
column-space of the dataset. These local relationships can provide insights for choosing and evaluating phase-out strategy effectiveness and resource allocation. 
More conventional metrics fall short in capturing the intricate relationships and nuances within this high-dimensional space.

To account for the complexity of the coal phaseout problem, we construct a graph model of the US coal fleet, using over 55 variables in the raw column space 
encompassing environmental, political, financial, and other variables. The graph and resulting groups arise from the structure of our high dimensional data set, 
whose complex relationships are distilled into a more interpretable model. Unlike traditional clustering methods, we use the graph model’s structure to build 
digestible descriptions of these groups. This allows us to illustrate which features connect the coal plants in each group, and compare group profiles across all 
available data fields.

\vspace{\baselineskip}

\highlight{How to interpret the model below:
\begin{itemize}
    \item \textit{Nodes} (circles) represent clusters of similar coal plants.
    \item \textit{Edges} (lines) connect similar clusters based on multidimensional relationships in the data.
    \item \textit{Groups} (connected nodes) make up connected components, or isolated sections of the graph.
\end{itemize}
}
%%%%%%%%%%%



%  ╭──────────────────────────────────────────────────────────╮
%  │ Coal Fleet Partition Fig                                 |
%  ╰──────────────────────────────────────────────────────────╯
\begin{figure}[H]
    \input{FIG-CoalFleetPartition}
    \caption{\textbf{Classifying the US Coal Fleet}}
    \medskip
    \footnotesize
    The resulting model has 8 unique groupings of coal plants. While our groupings are derived from every feature of the data set, looking at homogeneous data fields within each group gives a high-level and digestible overview of why certain plants are grouped together. Here, we label the groups based on some defining characteristics for increased interpretability.
    The node coloring corresponds to the percent of coal plants within each node with plans to retire. Nodes with a higher percentage of retiring coal plants are assigned a darker red. The size of each node is proportional to the total amount of carbon dioxide (CO\textsubscript{2}) emissions in 2022 from the coal plants within the node (scaled). 
    Larger nodes indicate a higher cumulative emission of CO\textsubscript{2} from the associated coal plants.
    \label{fig:coal-fleet-partition}
\end{figure}
%%%%%%%%%%%



\subsection{Understanding Tradeoffs}

\begin{figure}[H]
    \includesvg[inkscapelatex=false,width=\linewidth]{svg_figs/heatmap_small.svg}  
    \caption{\textbf{Plant Groups Comparison}}
    \medskip
    \footnotesize
    The heatmap illustrates the landscape of our coal plant groupings, delineating both the similarities and differences between groups. This helps to provide insights into potential barriers and incentives influencing retirement dynamics within each group.
    A subset of variables was chosen to aid in interpretability and readability, see SI.\ref{fig:heatmap-SIs} for full heatmap on all variables.
    \label{fig:heatmap}
\end{figure}

\subsubsection{Policy Efficacy and Optimizing Resource Allocation}



%  ╭──────────────────────────────────────────────────────────╮
%  │ Shortest Path Graphs Fig                                 |
%  ╰──────────────────────────────────────────────────────────╯
\begin{figure}[H]
    \centering
    \begin{minipage}{0.5\textwidth}
        \includesvg[inkscapelatex=false,width=\linewidth]{svg_figs/group3_path.svg}  
        \subcaption{\textbf{Group 3} - Distance to Sink Nodes}
    \end{minipage}%
    \begin{minipage}{0.5\textwidth}
        \includesvg[inkscapelatex=false,width=\linewidth]{svg_figs/group4_path.svg}  
        \subcaption{\textbf{Group 4} - Distance to Sink Nodes}
    \end{minipage}

    \begin{minipage}{0.25\textwidth}
        \includesvg[inkscapelatex=false,width=\linewidth]{svg_figs/group3_averageCashflow.svg}  
        \subcaption{Average Cashflows}
    \end{minipage}%
    \begin{minipage}{0.25\textwidth}
        \includesvg[inkscapelatex=false,width=\linewidth]{svg_figs/group3_PM2.5.svg}  
        \subcaption{PM 2.5 Emissions}
    \end{minipage}%
    \begin{minipage}{0.25\textwidth}
        \includesvg[inkscapelatex=false,width=\linewidth]{svg_figs/group4_ownership.svg}  
        \subcaption{Municipal Ownership}
    \end{minipage}%
    \begin{minipage}{0.25\textwidth}
        \includesvg[inkscapelatex=false,width=\linewidth]{svg_figs/group4_proCoal.svg}  
        \subcaption{Oppose Regulating Coal}
    \end{minipage}

    \caption{\textbf{Proximity To Retirement}}
    \medskip
    \footnotesize
    \textbf{(a, b)} Employing Dijkstra's algorithm, we determine the shortest path and corresponding distance from sink nodes to every other node in the graph. Nodes where all plants are planned for full retirement (100\% of plants within) are marked as sinks. These nodes are colored blue and labeled with an \textbf{S}. Distances signify the level of similarity; nodes with shorter paths to sinks share more similar attributes. Conversely, nodes with longer distances have more dissimilar characteristics, indicating a greater divergence between the plants contained within them and the attributes of the retiring plants within sink nodes.  This is our approximation for a \textit{coarse-grained} geodesic distance in high dimensional space. \textit{explain edge weighting here as well?}
    The edges in the graph are weighted to indicate the relationships between nodes. Nodes are sized according to the number of plants in each. 
    \textbf{(c, d, e, f)} Nodes are colored by variables that exhibit variance within the group, emphasizing inter-group distinctions. Higher values are assigned a darker red. The color scheme elucidates the spatial dynamics within connected components of the graph, revealing monotonic trends and differences in specific variables as distances from the source nodes increase.
    \label{fig:shortest-path-graphs}
\end{figure}



%%%%%%%%%%%

% Group 3:
% \begin{table}[H]
%     \centering
%     \small
%     \begin{adjustbox}{width=\textwidth}
%       \begin{tabular}{|l | l | l | l | l | l | l | l | l | l |}
%         \toprule
%         \textbf{Plant Name} & \textbf{State} & \textbf{Retirement Date} & \textbf{Ownership Type} & \textbf{Age at Retirement} & \textbf{Capacity Factor} & \textbf{Coal Nameplate (MW)} \\
%         \midrule\midrule
%         Limestone & TX & 2029 & Investor-Owned & 44 & 35\% & 1850 & \\
%         Belle River & MI & 2028 & Investor-Owned & 44 & 56\% & 1395 & \\
%         Welsh & TZ & 2028 & Investor-Owned & 51 & 44\% & 1116 & \\
%         \bottomrule
%     \end{tabular}
%   \end{adjustbox}
%   \caption{Dataset \& Data Sources Breakdown}
% \end{table}