
%  ╭────────────────────────────────────╮
%  │  curse of dimensionality           |
%  ╰────────────────────────────────────╯

\subsection{Coal and \textit{the Curse of Dimensionality}}

\jezTodo{Manifold Learning and Multiverse Analysis!}
The coal plant retirement problem suffers from \textit{the curse of dimensionality}, where as the dimensions of a dataset 
increase, the number of data points needed to guarantee reliable results grows exponentially. In the case of the current US coal 
fleet, we have a very limited number of coal plants with very high-dimensional data associated with each. As such, standard 
unsupervised methods fail to reliably predict which active coal plants have planned retirements on the books. Additionally, the 
current arsenal of artificial intelligence methods fall short when it comes to comparing and analyzing complex objects across different 
data disciplines. In regards to coal plants, we aim to analyze Environmental, Technical, Financial, Political, and Health data, making 
it again difficult to understand the complex interactions between all variables at a high level. 

Even advanced deep learning algorithms, 
such as Gradient Boosting, struggle to effectively overcome the curse of dimensionality and provide accurate results. When looking to 
understand the variables that are important in predicting whether a coal plant will retire, different deep learning methods give us 
different results, showing how complex the coal plant retirement problem really is (\textit{\textbf{See appendix for explanatory figs}}).

%  ╭────────────────────────────────────╮
%  │  what our approach brings          |
%  ╰────────────────────────────────────╯

\subsection{Advantages of a Novel Approach}\stuNote{Advantages and the Downfall of Manifold learning = we introduce multiverse analysis. This is where we highlight the ability to consider a distribution of graph models, and our selection criteria.}
We aim to acknowledge the transformational nature of coal phaseout motivations and propose an analytical framework that reflects the 
dynamic and multidimensional factors influencing coal plant retirement decisions. As such, we propose a novel method of analyzing the 
US coal plant space to provide new and actionable insight. Here is what our methodology adds to the academic discipline:

\begin{enumerate}
    \item \textbf{Contextual Flexibility:} By not relying on predefined composite scores or fixed weights, our method can adapt to different scenarios, regulatory environments, and stakeholder perspectives.
    \item \textbf{Holistic Understanding:} By allowing for a more nuanced and comprehensive understanding of a coal plant's/group of coal plants' performance, across various dimensions, we enable stakeholders to make more informed decisions.
    \item \textbf{Transparency:} Our approach makes the evaluation process more transparent by presenting data and relationships directly. This transparency can build trust among stakeholders, and help to avoid the interpretability issues inherent to composite scores.
    \item \textbf{Customization:} Different coal plants have unique characteristics. Our method can accommodate these variations, providing a more accurate representation of a group's strengths and weaknesses.
\end{enumerate}

We do not claim to have solved the coal plant retirement problem, nor to have figured out the best and fastest way to retire all coal plants. 
Instead, we bring a brand new perspective that gives policy makers and other stakeholders an in-depth understanding of the coal fleet from 
multiple perspectives.