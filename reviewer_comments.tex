Reviewer comments
Reviewer #1 (Remarks to the Author):

1. Key message:
The study presents an interesting perspective on coal plants phaseout in the US, where they show
primarily that basing retirement strategies on one parameter such as age is not suffcient, rather
they suggest that different coal plants require different strategies. They use a novel method in the
field, TDA, to group plants based on several criteria. Their results show that there is a trade-off
between high incentives and low barriers to phaseout. In other words, they show that the plants
that have low barriers to phaseout (for example: high costs) tend to have low high incentives to do
so (ex:low climate incentives).

2. What is notable about the study:
While introducing multidimensional phase out schedules is not new to the field, their use of the
TDA is very new and is quite fitting for the multidimensional aspect of the issue. They introduce
different relationships and highlight interesting trade-offs between the retirement criteria, that was
not uncovered before.

The paper digs deep into the different characteristics of the US coal fleet, and how trade-offs are unavoidable. Yet it lacks concrete real world implications, that if included would make the paper
an essential one in the field.
Results presented in the paper are interesting, I find the classification of phase-out barriers and incentives and the trade-offs between them very interesting. Yet, it is diffcult to follow the resultsclearly in the current manuscript, mainly due to the method (please see below).

Here are some suggestions and questions:

1. The paper introduces 16 clusters aggregated in 6 groups. The identication of clusters is not clear
for the reader.
- On what basis are these clusters identied? why 16 in specific?
- I appreciate the aggregation of the clusters into 6 groups to aid the readability, however, I think it still needs more detailed explanation. It is mentioned in pages 3 line 108 "that we aggregated these 16 clusters into six groups while maintaining a high degree of intra-group similarity across their defining characteristics". Yet this is not very clear, how were the aggregations done? Why is there a group called uncategorized? More details on aggregations and relating them to the spider plots in figure 2, would make the method and the results easier to follow.
- Some groups are not clearly understood from the name like high DAC impacts, low impacts. I suggest including a table that clearly denes each group and their corresponding clusters in the text or the SI is necessary to follow the paper.
- I could not follow what the DAC impacts group really means, are potential job losses included for such a group?, what sort of impact would phaseout have on those communities in terms of energy access, economic and social losses? I think addressing those questions clearly is essential.
- I am not sure how to understand the results as they are right now. Some clusters are larger than others and therefore their corresponding groups, and so it makes sense that these would be responsible for a higher share of generation and pollution. How do you address the different cluster sizes (and also groups) when interpreting your results, do you standardize by cluster/group size?


2. I do believe adding a correlation matrix/plot between the parameters for each group would aid the arguments presented in the paper, like for example for the statements on age in page 9 line 267-268.



Reviewer #2 (Remarks to the Author):

Overall comments:

This paper categorizes all the operating coal-fired power plants in the United States into 16 clusters and six large groups based on a set of technical, financial, environmental, political, and social criteria, using a method called the mapper algorithm. According to the grouping, it identifies different incentives and barriers to retire each group of coal plants, and further shows that there are trade-offs for prioritizing climate, social, or economic objectives. Overall, the manuscript is well-written and explores an important topic of developing tailored strategies for different coal plants’ retirement given multiple societal priorities, which would be of strong interest to both research and policy audiences; however, my main question is whether the multi-dimensional categorization of U.S. coal plants is a significant contribution to the field. I would expect to see additional analysis as a high-impact publication Nature Energy, for example, a retirement roadmap of the 2030/35 U.S. coal phaseout, and/or quantification of the multiple benefits and costs of the coal retirements, etc.

First, to elaborate my main comment, understanding coal retirement from multiple dimensions and across individual plants/units are not new in the literature [10, 16, 18]. The authors extend the analysis by introducing new social and political variables (i.e. DAC communities within 3 miles, and coal support) and by applying a new method (mapper algorithm). Aside from my questions about the methodology itself (see below), I am not convinced these incremental changes significantly advance existing literature that qualifies a publication in Nature Energy. I would like to ask the authors to further explore a (or alternative) retirement roadmap based on findings of the plant grouping. Since all U.S. coal plants need to be closed by 2030/35 for meeting its climate target, some of the key questions include the retirement schedule and the broad social and economic implications. These questions are already being investigated in the literature [8-10, 16], which I think the authors can meaningfully add to using the insights gained from the mapper analysis.

Second, it is unclear whether the unit of analysis is a coal plant or a generator. It seems the authors compile generator-level data, but the grouping is conducted based on plant average. Please clarify. For plants that contain multiple generators, it is likely that they vary largely regarding the non-location based criteria, and the analysis based on plant average could be misleading. It is also common to close a plant generator by generator, not necessarily all units together.

Third, it is also unclear to me how the 16 clusters are further combined into 6 groups. Specifically, two large groups – large env. Impacts and high DAC impacts – have multiple clusters. How are they identified as the key factor that is shared across several clusters but differentiated with other groups? For example, looking at the box plots for cluster o under the high DAC impacts group (Fig 2a), the mean and distribution of forward costs, affected DAC, and public supports are clearly different than the other two clusters (g & b) under the same group. It would be helpful to provide more description for readers unfamiliar with the mapper algorithm.

Last but not least, I have several questions regarding specific variables.
1) The variable going-forward costs comes from the coal cost crossover dataset [33], a secondary source that is not peer-reviewed. The original data is to compare the coal plants with the nearby renewables and to show that which coal plants can be replaced by cost-competitive alternatives, which is more important in determining plant closure. Instead, the authors use it to compare among coal plants themselves; however, a coal plant cheaper to run may locate in a place with abundant RE resources and have lower barrier for retirement.
2) In terms of the impacts on vulnerable and disadvantaged communities, simply using the number within 3 miles might be misleading, especially when they find plants in this group tend to be small with very low generation. The employment, economic, and maybe health impacts, either direct or indirect, are related to plant size, generation, and how important they are in the local economy. Without taking into account those factors, the impacts of the DAC group can be overestimated.
3) All variables focus on plant total (i.e. generation, SO2 emissions, pollution impact), which is highly correlated to plant size. I would suggest adding variables on a per unit basis (of generation, or capacity) to reflect plant efficiency.


Specific comments:

1) Page 12, how plantPOP is calculated?
2) The authors may consider using different colors for incentives vs. barriers in the figures.
3) The group name “large environmental impact” is not accurate. May consider changing it to “large public health impact”.



Reviewer #3 (Remarks to the Author):

Thank you for the opportunity to review. Decarbonization and decarbonization policy are hugely important and of great interest, and I’m always glad to see detailed attention to coal. Although the research focus is interesting, I am not recommending publication of this manuscript at this time, primarily because of a lack of validation on the attributes driving retirements and a lack of precision on some of the input data that in my view have some qualitatively important impacts on the results. The manuscript seems to be unclear on whether it is proposing a new prioritization schema for coal retirements (though without a deep discussion of the extremely urgent timelines, and with multiple references that seem to indicate the authors find prioritization schemas not to especially useful) versus a cluster analysis grouping plants with predictive retirement attributes that can be used to support decisions, or similar. The methods do not clearly distinguish between plant and generator level analyses, and some comments raise concerns about whether the data were appropriately identified and cleaned (the methods/SI are not detailed enough to evaluate). I also felt the environmental justice elements were overly simplified and overlook the extensive impacts to communities that are further away from plants, which is the topic of a pretty extensive air emissions modeling literature. Although some of these issues could be addressed with revision, I overall found the piece to be interesting but very theoretical (particularly since the evaluated attributes were not demonstrated to be correlated with/predictive of retirements or disproportionately valuable policy outcomes, especially on the short time scale associated with US coal retirement needs/targets, and actual US coal policy was not really discussed — limiting the value of the study for informing practice), and might fit better in a journal with a more methodological focus — as the authors point out, the method is interesting and could be widely applied in other contexts.

Abstract:
Suggest adding a couple of numbers up front to help contextualize — for the plants in DACs, what percentage of facilities vs. percentage of gen/CO2 is this, for example? And similarly for the highly utilized ones. I think it would be more helpful to the reader to include the value (e.g., “60%”) vs. using terms like “lion’s share” or “immense”
L22: clarify “retirement-age based” here — given the motivation at the beginning that plants need to be phased out by 2030-2035, is the argument that that motivation is wrong? or something else? retirement age wasn’t introduced earlier, I think

Intro
L25 — reducing, or eliminating?
L28 — is this CO2 only, or all GHGs? just checking that this is the metric you’re referencing (if GHGs, also recommend stating the GWP used for CH4 as that will affect the value at 2 sig figs)
L33 — compared to NGGTs, right? NGCCs / NGSTs aren’t that flexible either. suggest a citation here — it’s an important point but needs to be made precisely and with a reference (the EIA ramp rate data might be a good one)
L34 — I think “dominated” by variable renewable energy needs a citation
L35 — how are you defining stranded?
L37 — coal phaseout, or coal power phase out?
L43 — “actually slated” — I’d suggest “currently announced” or similar. by the time coal plants submit a retirement date to EIA it’s usually pretty deep in the process (maybe a year or two out), and I wouldn’t read those announcements as a correct population survey of retirements by 2030. that is: the EIA reports are not a good indicator of all retirements that will happen by 2030. agreed that with essentially no policy on this outside of Illinois under CEJA something else needs to happen beyond just letting retirements play out, but I think the manuscript is overstating the certainty about plants remaining online.
L54 — operational lifespans are not deterministic, but rather, averages. some plants close before that, some after — I think it’s important to be clear that this is not an expected lifetime so much as a historically observed mean age on retirement. “expected lifetime” sometimes refers to book life, which is more like 30 years, too. distinguishing between useful life, typical retirement age, book life, etc. might be too much for this piece but I don’t think it’s right to say that plant age is a retirement criterion, but rather an observed outcome variable (that happens to be pretty stable at a population level even if it’s not the thing that actually causes retirement)
L65 — strongly suggest defining what you mean by “early retirement” here, particularly given the point that many plants are operating beyond a typical lifespan. is your analysis only focused on the plants that are less than a certain age by 2030?
L66 — I get the point about variability, but in practice, we’re talking about a retirement timeline of ~7 years per the motivation of the study — is there even space for prioritization at this point, vs. just a closure requirement? if this is more about “we might miss 2030 but here’s how something can get done,” suggest including some background to that effect as well. (same comment about “early” applies throughout)
L84 — if you’re using 1 coal EGU as the determinant of eligibility, suggest the more precise term of “power plants including coal generation” over “coal power plants” here — and/or including detail on how many of these plants are predominately fueled by something else. also: suggest clarifying up here whether all those attributes are for the plant vs. the collection of coal EGUs at the plant. relatedly: are you only looking at electricity sector plants, or does this include commercial and industrial EGUs, which have very different profiles?
L88 — retrofit costs for what? also: one of the most important financial / debt drivers of plant closure is recent investment, and specifically \$100m+ investments in FGD in response to MATS. strongly suggest either including recent retrofit activity (one proxy variable is a step change in SOx emissions) or discussing this in detail as a limitation.
L96 — “absent policy,” right?
L100 — I don’t think this certainly follows from the attributes used in the study. ongoing debt service requirements, ownership, role in power system reliability, compliance issues (e.g., with CCR, MATS/CSAPR, 316b, state laws) and PUC / regulatory setting are in my experience a lot more important to retirement decisions (absent political / regulatory intervention) than things like DAC proximity, pollutants (as long as they’re below legal limits), retrofit costs. It’s not clear how the attributes were chosen, and how they were validated as retirement drivers — and in particular given the colinearity of a lot of these (e.g., gen, capacity factor, SO2, pollutant impacts, fuel costs — that are all driven by coal combustion volume, basically), it’s not clear that these are the right retirement drivers. the study claims to be circumventing flawed ways of prioritizing retirements but it seems that this also a set of general prioritization factors that are not demonstrably tied to retirement incentives/barriers / shown to be the most important independent drivers, at least in the text.
L122 — are these plants in compliance with federal pollution laws?
L155 — make the link between disproportionate generation and environmental impact -- the logic of this sentence was confusing
page 7 was missing in the PDF I received
L207 — agreed, but it’s not clear that the analysis here actually addresses barriers and incentives as they exist in practice. the attributes identified here are important and relevant but not proven to be linked to actual retirement decisions
L217 — I’m totally sold on the point that reitrement plans need to look different for different plants, but I think the manuscript fails to deal with the timeline issue — the MS points out the 2030-2035 timeline numerous times but does not really grapple with the point that this is in ~10 years. what does this evaluation help with in practice? do we actually have time to do anything but a CEJA-style retirement deadline? if we don’t, what are the tradeoffs?
L224 — I recognize we’re in a totally different context now but I think it’s important to acknowledge how much coal has retired in the last 10-15 years in the US, and how much local pollution exposure has resulted from those closures (and FGD installations). we’re running out of plants with huge and obvious pollution mitigation opportunities in part precisely because this has been as successful as it has — we’re not starting from the stable baseline, but rather a clearly and consistently declining industry that has been responding to these forces for a long time. this is also partly why I think the manuscript needs to much more clearly defend the attributes it uses as causal / predictive as retirement barriers/incentives — we have 100s of GW of empirical data about coal EGU closures from the recent US past that can be used to validate these kinds of claims, or at least be used to contextualize arguments about what has changed and why the drivers are different now
L235 — I’m surprised to see this as one of the main results takeaways, as I didn’t think this was heavily emphasized. Also, the notion that DACs within 3 miles are the main drivers of a justice-focused phaseout is surprising, given the very long distances over which health impacts are felt — see, e.g., Lucas Henneman’s work
L243 — again, I don’t think this is necessarily true based only on population sparseness. stacks are tall for a reason. also, EJ issues from coal /extraction/ can be significant and are not usually colocated with plants.
L264 — this study has not shown that its attributes are more successful retirement predictors than EGU age, so this is a surprising conclusion — especially given the lack of discussion of limitations. per L268 — I don’t think it’s a particularly common argument that older plants are the highest priority for retirement under various value systems, but rather that plant age is a decent empirical variable that is highly correlated with observed retirements. in any case, though, again we’re talking about the difference of a few years in a 2030-2035 context, so it’s unclear how much this matters.
L284 — eGRID reports generator level emissions for many, though not all, EGUs. generation as a CO2 proxy is a bit of a surprising choice given pretty extensive prior work on this. also, EIA does provide generator level data that can be used to find heat rate, which can be used as a test for efficiency impacts on CO2 emissions. also: did you distinguish among the multiple types of coal being burned in the EGUs, which have very different emissions profiles? were the non-CO2 impacts proxied via generation as well, or taken from a specific data source?
Table 1 — suggest clarifying the resolution here as well. what is EGU, what is plant?
recognizing that 923 is a beautifully chaotic resource with a lot of data — I’m quite familiar with 923 and can’t tell what you are using as “retrofit costs” from that resource. retrofit for what, and what’s the page/column?
